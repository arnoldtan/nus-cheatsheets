\documentclass[../ma2001_notes.tex]{subfiles}

\begin{document}
\chapter{Linear Transformation}
\section{Linear Transformations from $\mathbb{R}^n$ to $\mathbb{R}^m$}
\subsection{Definition of Linear Transformation}
\textbf{Definition.} The mapping \(T:\mathbb{R}^n\to\mathbb{R}^m\) defined by
\[T\left(\begin{pmatrix}
	x_1 \\ x_2 \\ \vdots \\ x_n
\end{pmatrix}\right)=\left(
\arraycolsep=1.4pt\begin{array}{ccccccc}
	a_{11}x_1 & + & a_{12}x_2 & + & \cdots & + & a_{1n}x_n \\
	a_{21}x_1 & + & a_{22}x_2 & + & \cdots & + & a_{2n}x_n \\
	& \vdots & & \vdots & & \vdots & \\
	a_{m1}x_1 & + & a_{m2}x_2 & + & \cdots & + & a_{mn}x_n \\
\end{array}\right)\]
a \textbf{linear transformation} from \(\mathbb{R}^n\) to \(\mathbb{R}^m\).

\subsubsection{Linear Operator}
\(T\) is a \textbf{linear operator} on \(\mathbb{R}^n\) if \(m=n\).

\subsection{Linear Transformation As Matrix Form}
A \textbf{linear transformation} is viewed as the \textbf{matrix form}:
\[T\left(\begin{pmatrix}
	x_1 \\ x_2 \\ \vdots \\ x_n
\end{pmatrix}\right)=\begin{pmatrix}
	a_{11} & a_{12} & \cdots & a_{1n} \\
	a_{21} & a_{22} & \cdots & a_{2n} \\
	\vdots & \vdots & \vdots & \vdots \\
	a_{m1} & a_{m2} & \cdots & a_{mn} \\
\end{pmatrix}\begin{pmatrix}
	x_1 \\ x_2 \\ \vdots \\ x_n
\end{pmatrix}\]
\(T:\mathbb{R}^n\to\mathbb{R}^m\) such that \(\forall\bm{x}\in\mathbb{R}^n\quad T(\bm{x})=\bm{Ax}\)
\begin{itemize}
	\item\(\bm{A}=(a_{ij})_{m\times n}\) is the \textbf{standard matrix} for \(T\).
\end{itemize}

\subsection{Examples of Linear Transformation}
\subsubsection{Identity Transformation}
\textbf{Definition.} Let \(I:\mathbb{R}^n\to\mathbb{R}^n\) be the \textbf{linear transformation} \(\forall\bm{x}\in\mathbb{R}^n\quad I(\bm{x})=\bm{x}\).
\begin{itemize}
	\item It is the
	\begin{itemize}
		\item\textbf{identity transformation}; or
		\item\textbf{identity operator} on \(\mathbb{R}^n\).
	\end{itemize}
	\item\(I(\bm{x})=\bm{x}=\bm{I}_n\bm{x}\Rightarrow\bm{I}_n\) is the \textbf{standard matrix} for \(I\).
\end{itemize}

\subsubsection{Zero Transformation}
\textbf{Definition.} Let \(O:\mathbb{R}^n\to\mathbb{R}^m\) be the \textbf{linear transformation} \(\forall\bm{x}\in\mathbb{R}^n\quad O(\bm{x})=\bm{0}\).
\begin{itemize}
	\item It is the \textbf{zero transformation}.
	\item\(O(\bm{x})=\bm{0}=\bm{0}_{m\times n}\bm{x}\Rightarrow\bm{0}_{m\times n}\) is the \textbf{standard matrix} for \(O\).
\end{itemize}

\subsection{Standard Matrix of a Linear Transformation is Unique}
\textbf{Theorem.} The \textbf{standard matrix} of a \textbf{linear transformation} is unique.

\subsection{To Prove that a Mapping is a Linear Transformation}
To show that \(T:\mathbb{R}^n\to\mathbb{R}^m\) is a \textbf{linear transformation}, it suffices to find a \textbf{matrix} \(\bm{A}\) so that \(\forall\bm{x}\in\mathbb{R}^n\quad T(\bm{x})=\bm{Ax}\).

\subsection{Linearity of Linear Transformations}
\textbf{Theorem.} Let \(T:\mathbb{R}^n\to\mathbb{R}^m\) be a \textbf{linear transformation}. Then
\begin{itemize}
	\item\(T(\bm{0})=\bm{0}\)
	\item\(\forall c\in\mathbb{R},\forall\bm{v}\in\mathbb{R}^n,\quad T(c\bm{v})=cT(\bm{v})\)
	\item\(\forall\bm{u},\bm{v}\in\mathbb{R}^n,\quad T(\bm{u}+\bm{v})=T(\bm{u})+T(\bm{v})\)
	\item\(\forall c_1,\ldots,c_k\in\mathbb{R},\forall\bm{v}_1,\ldots,\bm{v}_k\in\mathbb{R}^n,\quad T(c_1\bm{v}_1+\cdots+c_k\bm{v}_k)=c_1T(\bm{v}_1)+\cdots+c_kT(\bm{v}_k)\)
\end{itemize}

\subsection{To Prove that a Mapping is Not a Linear Transformation}
To show that a mapping \(T:\mathbb{R}^n\to\mathbb{R}^m\) is \textbf{not} a \textbf{linear transformation},
\begin{itemize}
	\item Show that \(T(\bm{0})\ne\bm{0}\); or
	\item Find \(c\in\mathbb{R},\bm{v}\in\mathbb{R}^n\) such that \(T(c\bm{v})\ne cT(\bm{v})\); or
	\item Find \(\bm{u},\bm{v}\in\mathbb{R}^n\) such that \(T(\bm{u}+\bm{v})\ne T(\bm{u})+T(\bm{v})\).
\end{itemize}

\subsection{Representation of Linear Transformations}
\subsubsection{Linear Transformations are Completely Determined by a Basis of $\mathbb{R}^n$ and its Images}
Suppose that \(T:\mathbb{R}^n\to\mathbb{R}^m\) is a \textbf{linear transformation} and \(S=\{\bm{v}_1,\ldots,\bm{v}_n\}\) is a \textbf{basis} for \(\mathbb{R}^n\). Then \(\forall\bm{v}\in\mathbb{R}^n\quad(\bm{v})_S=(c_1,\ldots,c_n)\), and
\begin{align*}
	T(\bm{v})
	&=T(c_1\bm{v}_1+\cdots+c_n\bm{v}_n) \\
	&=c_1T(\bm{v}_1)+\cdots+c_nT(\bm{v}_n)
\end{align*}
\(T(\bm{v})\) is completely determined by \(T(\bm{v}_1),\ldots,T(\bm{v}_n)\). Let
\begin{itemize}
	\item\(\bm{A}\) be the \textbf{standard matrix} for \(T\),
	\item\(\bm{B}=\begin{pmatrix}
		T(\bm{v}_1) & \cdots & T(\bm{v}_n)
	\end{pmatrix}\), and
	\item\(\bm{P}=\begin{pmatrix}
		\bm{v}_1 & \cdots & \bm{v}_n
	\end{pmatrix}\)
\end{itemize}
Then \(\bm{A}=\bm{BP}^{-1}\).

\subsubsection{Columns of Standard Matrix are Images of Standard Basis of $\mathbb{R}^n$}
Let \(T:\mathbb{R}^n\to\mathbb{R}^m\) be a \textbf{linear transformation} and \(\bm{A}\) be the \textbf{standard matrix} for \(T\). Then \(\bm{A}=\begin{pmatrix}
	T(\bm{e}_1) & \cdots & T(\bm{e}_n)
\end{pmatrix}\).

\subsection{Linear Combination Preserving Implies Linear Transformation}
\subsubsection{With \(k\) vectors}
\textbf{Theorem.} A mapping \(T:\mathbb{R}^n\to\mathbb{R}^m\) is a \textbf{linear transformation}, i.e., \(T(\bm{x})=\bm{Ax}\Leftrightarrow\forall c_1,\ldots,c_k\in\mathbb{R},\forall\bm{v}_1,\ldots,\bm{v}_k\in\mathbb{R}^n,\quad T(c_1\bm{v}_1+\cdots+c_k\bm{v}_k)=c_1T(\bm{v}_1)+\cdots+c_kT(\bm{v}_k)\)

\subsubsection{With Two Vectors}
\textbf{Theorem.} A mapping \(T:\mathbb{R}^n\to\mathbb{R}^m\) is a \textbf{linear transformation}, i.e., \(T(\bm{x})=\bm{Ax}\Leftrightarrow\forall c,d\in\mathbb{R},\forall\bm{u},\bm{v}\in\mathbb{R}^n,\quad T(c\bm{u}+d\bm{v})=cT(\bm{u})+dT(\bm{v})\).

\subsection{General Definition of Linear Transformations}
\textbf{Definition.} Let \(V\) and \(W\) be \textbf{vector spaces}. A mapping \(T:V\to W\) is a \textbf{linear transformation} if
\[\forall\bm{u},\bm{v}\in\mathbb{R}^n,\forall c,d,\in\mathbb{R},\quad T(c\bm{u}+d\bm{v})=cT(\bm{u})+dT(\bm{v})\]

\subsection{Change of Bases}
\subsubsection{Change Between the Images of Two Bases}
\textbf{Theorem.} Let \(T:\mathbb{R}^n\to\mathbb{R}^m\) be a \textbf{linear transformation}. Let
\begin{itemize}
	\item\(S=\{\bm{u}_1,\ldots,\bm{u}_n\}\) and \(R=\{\bm{v}_1,\ldots,\bm{v}_n\}\) be \textbf{bases} for \(\mathbb{R}^n\),
	\item\(\bm{B}=\begin{pmatrix}
	T(\bm{u}_1) & \cdots & T(\bm{u}_n)
\end{pmatrix}\) and \(\bm{C}=\begin{pmatrix}
	T(\bm{v}_1) & \cdots & T(\bm{v}_n)
\end{pmatrix}\), and
	\item\(\bm{P}\) be the \textbf{transition matrix} from \(S\) to \(R\) and \(\bm{Q}=\bm{P}^{-1}\)
\end{itemize}
Then \(\bm{B}=\bm{CP}\) and \(\bm{C}=\bm{BQ}\).

\subsubsection{Change Basis for Linear Operator}
\textbf{Theorem.} Let \(T:\mathbb{R}^n\to\mathbb{R}^n\) be a \textbf{linear operation} on \(\mathbb{R}^n\) and \(\bm{A}\) be the \textbf{standard matrix}. Let \(S=\{\bm{v}_1,\ldots,\bm{v}_n\}\) be a \textbf{basis} for \(\mathbb{R}^n\) and \(\bm{P}=\begin{pmatrix}
	\bm{v}_1 & \cdots & \bm{v}_n
\end{pmatrix}\). Then
\begin{itemize}
	\item\([T(\bm{v})]_S=\bm{B}[\bm{v}]_S\), where \(\bm{B}=\bm{P}^{-1}\bm{AP}\)
	\item\(T\) can be represented by \([\bm{v}]_S\mapsto\bm{B}[\bm{v}]_S\)
	\item\(\bm{A}\) and \(\bm{B}\) are \textbf{similar}.
\end{itemize}

\subsection{Composition}
\subsubsection{Definition of Function Composition}
\textbf{Definition.} Let \(f:X\to Y\) and \(g:Y\to Z\) be two functions. The \textbf{composition} of \(g\) with \(f\), denoted by \(g\circ f\), is the function \(X\to Z\) such that \(\forall x\in X\quad (g\circ f)(x)=g(f(x))\)

\subsubsection{Function Composition is not Commutative in General}
Note: In general, \(g\circ f\ne f\circ g\).

\subsubsection{Linear Transformation Composition}
\textbf{Definition.} Let \(S:\mathbb{R}^n\to\mathbb{R}^m\) and \(T:\mathbb{R}^m\to\mathbb{R}^k\) be \textbf{linear transformations}. The \textbf{composition} of \(T\) with \(S\), denoted by \(T\circ S\), is the mapping \(\mathbb{R}^n\to\mathbb{R}^k\) such that \(\forall\bm{u}\in\mathbb{R}^n,\quad(T\circ S)(\bm{u})=T(S(\bm{u}))\)

\subsubsection{Linear Transformation Composition is not Commutative in General}
Note: In general, \(T\circ S\ne S\circ T\).

\subsubsection{Composition of Linear Transformations is a Linear Transformation and Formula for its Standard Matrix}
\textbf{Theorem.} Let \(S:\mathbb{R}^n\to\mathbb{R}^m\) and \(T:\mathbb{R}^m\to\mathbb{R}^k\) be \textbf{linear transformations}. Then
\begin{itemize}
	\item\(T\circ S:\mathbb{R}^n\to\mathbb{R}^k\) is a \textbf{linear transformation}
\end{itemize}
Let \(\bm{A}\) and \(\bm{B}\) be the \textbf{standard matrices} for \(S\) and \(T\) respectively. Then
\begin{itemize}
	\item the \textbf{standard matrix} for \(T\circ S\) is \(\bm{BA}\).
\end{itemize}

\subsubsection{Properties of Linear Transformation Composition}
\textbf{Theorem.} Let \(S,T,T_1,T_2,S_1,S_2,U\) be \textbf{linear transformations} and \(c\in\mathbb{R}\). Then
\begin{itemize}
	\item\textbf{Identity \& Zero Composition:} \(I\circ S=S\circ I=S\); \(O\circ S=S\circ O=O\)
	\item\textbf{Constant Multiple:} \(c(T\circ S)=(cT)\circ S=T\circ(cS)\)
	\item\textbf{Associativity:} \(U\circ(T\circ S)=(U\circ T)\circ S\)
	\item\textbf{Distributive Over Addition:}
	\begin{itemize}
		\item\((T_1+T_2)\circ S=T_1\circ S+T_2\circ S\)
		\item\(T\circ(S_1+S_2)=T\circ S_1+T\circ S_2\)
	\end{itemize}
\end{itemize}

\section{Ranges and Kernels}
\subsection{Range}
\subsubsection{Range of Function}
\textbf{Definition.} Let \(f:X\to Y\) be a \textbf{function}. The \textbf{range} of \(f\), denoted by \(\R(f)\), is the set of all \textbf{images} of \(f\):
\[\R(f)=\{f(x)\mid x\in X\}\subseteq Y\]

\subsubsection{Range of Linear Transformation}
\textbf{Definition.} Let \(T:\mathbb{R}^n\to\mathbb{R}^m\) be a \textbf{linear transformation}. The \textbf{range} of \(T\), denoted by \(\R(T)\), is the set of all \textbf{images} of \(T\):
\[\R(T)=\{T(\bm{v})\mid\bm{v}\in\mathbb{R}^n\}\subseteq\mathbb{R}^m\]

\subsubsection{Representation of Range}
\paragraph{Range as Subspace Spanned by the Images of Any Basis for $\mathbb{R}^n$}\,\\
\textbf{Theorem.} Let \(T:\mathbb{R}^n\to\mathbb{R}^m\) be a \textbf{linear transformation} and \(\{\bm{v}_1,\ldots,\bm{v}_n\}\) be any basis for \(\mathbb{R}^n\). Then
\begin{itemize}
	\item\(\R(T)=\vspan\{T(\bm{v}_1),\ldots,T(\bm{v}_n)\}\).
	\item\(\R(T)\) is a \textbf{subspace} of \(\mathbb{R}^m\).
\end{itemize}

\paragraph{Range as Column Space of Standard Matrix}\,\\
\textbf{Theorem.} Let \(T:\mathbb{R}^n\to\mathbb{R}^m\) be a \textbf{linear transformation} and \(\bm{A}\) be the \textbf{standard matrix} for \(T\). Then \(\R(T)=\) \textbf{column space} of \(\bm{A}\).

\subsection{Rank of Linear Transformation}
\textbf{Definition.} Let \(T\) be a \textbf{linear transformation}. The \textbf{rank} of \(T\), denoted by \(\rank(T)\), is defined as
\[\rank(T)=\dim(\R(T))\]
Let \(\bm{A}\) be the \textbf{standard matrix} for \(T\). Then
\[\rank(T)=\dim(\R(T))=\dim(\textbf{column space}\text{ of }\bm{A})=\rank(\bm{A})\]

\subsection{Kernel}
\subsubsection{Definition of Kernel of Linear Transformation}
\textbf{Definition.} Let \(T:\mathbb{R}^n\to\mathbb{R}^m\) be a \textbf{linear transformation}. The \textbf{kernel} of \(T\), denoted by \(\Ker(T)\), is
\[\Ker(T)=\{\bm{v}\in\mathbb{R}^n\mid T(\bm{v})=\bm{0}\}\subseteq\mathbb{R}^n\]
Recall that \(T(\bm{0})=\bm{0}\Rightarrow\bm{0}\in\Ker(T)\subseteq\mathbb{R}^n\).

\subsubsection{Representation of Kernel}
\paragraph{Kernel as Nullspace of Standard Matrix and Subspace of $\mathbb{R}^n$}\,\\
\textbf{Theorem.} Let \(T:\mathbb{R}^n\to\mathbb{R}^m\) be a \textbf{linear transformation} and \(\bm{A}\) be the \textbf{standard matrix} for \(T\). Then \(\Ker(T)=\) (\textbf{nullspace} of \(\bm{A}\)) which is a \textbf{subspace} of \(\mathbb{R}^n\).

\paragraph{Finding General Solution for Kernel}\,\\
Let \(T:\mathbb{R}^n\to\mathbb{R}^m\) be a \textbf{linear transformation} and \(\bm{A}\) be the \textbf{standard matrix} for \(T\). To find a general solution for \(\Ker(T)\), solve the linear system \(\bm{Ax}=\bm{0}\).

\subsection{Nullity of Linear Transformation}
\textbf{Definition.} Let \(T\) be a \textbf{linear transformation}. The \textbf{nullity} of \(T\), denoted by \(\nullity(T)\), is defined as
\[\nullity(T)=\dim(\Ker(T))\]
Let \(\bm{A}\) be the \textbf{standard matrix} for \(T\). Then
\[\nullity(T)=\dim(\Ker(T))=\dim(\textbf{nullspace}\text{ of }\bm{A})=\nullity(\bm{A})\]

\subsection{Dimension Theorem for Linear Transformations}
\textbf{Theorem.} Let \(T:\mathbb{R}^n\to\mathbb{R}^m\) be a \textbf{linear transformation}. Then \(\rank(T)+\nullity(T)=n\).

\subsection{Linear Transformation Between Vector Spaces}
Let \(T:V\to W\) be a \textbf{linear transformation} between \textbf{vector spaces}.

\subsubsection{Range}
\textbf{Definition.} \(\R(T)=\{T(\bm{v}\mid\bm{v}\in V\}\) is a \textbf{subspace} of \(W\).

\subsubsection{Rank}
\textbf{Definition.} \(\rank(T)=\dim(\R(T))\).

\subsubsection{Kernel}
\textbf{Definition.} \(\Ker(T)=\{\bm{v}\in V\mid T(\bm{v})=\bm{0}\}\) is a \textbf{subspace} of \(V\).

\subsubsection{Nullity}
\textbf{Definition.} \(\nullity(T)=\dim(\Ker(T))\).

\subsubsection{Dimension Theorem}
\textbf{Definition.} \(\rank(T)+\nullity(T)=\dim(V)\).


\section{Geometric Linear Transformations}
\subsection{Geometric Interpretation Completely Determined by Effect on a Basis for its Domain}
Since the images of any \textbf{basis} for \(\mathbb{R}^n\) completely determines a \textbf{linear transformation} \(T:\mathbb{R}^n\to\mathbb{R}^m\), to study the geometric interpretation of a \textbf{linear transformation}, it suffices to check the effect of the \textbf{linear transformation} on a \textbf{basis} for its domain.

\subsection{Scaling}
\subsubsection{Scaling in $\mathbb{R}^2$}
Let \(T:\mathbb{R}^2\to\mathbb{R}^2\) be a \textbf{linear transformation} and \(\begin{pmatrix}
	\lambda_1 & 0 \\ 0 & \lambda_2
\end{pmatrix}\) be the \textbf{standard matrix} for \(T\), where \(\lambda_1,\lambda_2>0\).
\[T\left(\begin{pmatrix}
	x \\ y
\end{pmatrix}\right)=\begin{pmatrix}
	\lambda_1 & 0 \\ 0 & \lambda_2
\end{pmatrix}\begin{pmatrix}
	x \\ y
\end{pmatrix}=\begin{pmatrix}
	\lambda_1x \\ \lambda_2y
\end{pmatrix}\]
Then \(T\) is a \textbf{scaling} in \(\mathbb{R}^2\),
\begin{itemize}
	\item along the \(x\)-axis by a factor of \(\lambda_1\), and
	\item along the \(y\)-axis by a factor of \(\lambda_2\).
\end{itemize}

\paragraph{Dilation \& Contraction}\,\\
Suppose further that \(\lambda_1=\lambda_2\). Let \(\lambda=\lambda_1=\lambda_2\). Then
\begin{itemize}
	\item\(T\) is a \textbf{dilation} if \(\lambda>1\).
	\item\(T\) is a \textbf{contraction} if \(0<\lambda<1\).
\end{itemize}

\subsubsection{Scaling in other Axes in $\mathbb{R}^2$}
Let \(T:\mathbb{R}^2\to\mathbb{R}^2\) be a \textbf{linear transformation} and \(\bm{A}\) be the \textbf{standard matrix}.
\begin{itemize}
	\item Suppose \(\bm{A}\) is \textbf{diagonalizable}, say \(\exists\bm{P}\) such that \(\bm{P}^{-1}\bm{AP}=\begin{pmatrix}
		\lambda_1 & 0 \\ 0 & \lambda_2
	\end{pmatrix},\lambda_1,\lambda_2>0\)
	\item Suppose \(\bm{P}=\begin{pmatrix}
		\bm{v}_1 & \bm{v}_2
	\end{pmatrix}\). Let \(S=\{\bm{v}_1,\bm{v}_2\}\). Then \(S\) is a \textbf{basis} for \(\mathbb{R}^2\).
	\item\(\forall\bm{v}\in\mathbb{R}^2,\quad[T(\bm{v})]_S=\begin{pmatrix}
		\lambda_1 & 0 \\ 0 & \lambda_2
	\end{pmatrix}[\bm{v}]_S\) (from section 7.1.11.2)
	\item Then \(T\) can be viewed as a scaling
	\begin{itemize}
		\item along the direction of \(\bm{v}_1\) by factor \(\lambda_1\),
		\item along the direction of \(\bm{v}_2\) by factor \(\lambda_2\),
	\end{itemize}
\end{itemize}

\subsubsection{Scaling in $\mathbb{R}^3$}
Let \(T:\mathbb{R}^3\to\mathbb{R}^3\) be a \textbf{linear transformation} with \textbf{standard matrix} \(\begin{pmatrix}
	\lambda_1 & 0 & 0 \\
	0 & \lambda_2 & 0 \\
	0 & 0 & \lambda_3
\end{pmatrix}\), where \(\lambda_1,\lambda_2,\lambda_3>0\).
\[T\left(\begin{pmatrix}
	x \\ y \\ z
\end{pmatrix}\right)=\begin{pmatrix}
	\lambda_1 & 0 & 0 \\
	0 & \lambda_2 & 0 \\
	0 & 0 & \lambda_3
\end{pmatrix}\begin{pmatrix}
	x \\ y \\ z
\end{pmatrix}=\begin{pmatrix}
	\lambda_1x \\ \lambda_2y \\ \lambda_3z
\end{pmatrix}\]
Then \(T\) is a \textbf{scaling},
\begin{itemize}
	\item along the \(x\)-axis by a factor of \(\lambda_1\),
	\item along the \(y\)-axis by a factor of \(\lambda_2\),
	\item along the \(z\)-axis by a factor of \(\lambda_3\).
\end{itemize}


\paragraph{Dilation \& Contraction}\,\\
Suppose further that \(\lambda_1=\lambda_2=\lambda_3=\lambda\). Then
\begin{itemize}
	\item\(T\) is a \textbf{dilation} if \(\lambda>1\).
	\item\(T\) is a \textbf{contraction} if \(0<\lambda<1\).
\end{itemize}

\subsubsection{Scaling in other Axes in $\mathbb{R}^3$}
Let \(T:\mathbb{R}^3\to\mathbb{R}^3\) be a \textbf{linear transformation} with \textbf{standard matrix} \(\bm{A}\).
\begin{itemize}
	\item Suppose \(\bm{A}\) is \textbf{diagonalizable}, say \(\exists\bm{P}\) such that \(\bm{P}^{-1}\bm{AP}=\begin{pmatrix}
	\lambda_1 & 0 & 0 \\
	0 & \lambda_2 & 0 \\
	0 & 0 & \lambda_3
\end{pmatrix},\lambda_1,\lambda_2,\lambda_3>0\)
	\item Suppose \(\bm{P}=\begin{pmatrix}
		\bm{v}_1 & \bm{v}_2 & \bm{v}_3
	\end{pmatrix}\). Let \(S=\{\bm{v}_1,\bm{v}_2,\bm{v}_3\}\). Then \(S\) is a \textbf{basis} for \(\mathbb{R}^3\).
	\item\(\forall\bm{v}\in\mathbb{R}^3,\quad[T(\bm{v})]_S=\begin{pmatrix}
	\lambda_1 & 0 & 0 \\
	0 & \lambda_2 & 0 \\
	0 & 0 & \lambda_3
\end{pmatrix}[\bm{v}]_S\) (from section 7.1.11.2)
	\item Then \(T\) can be viewed as a scaling
	\begin{itemize}
		\item along the direction of \(\bm{v}_1\) by factor \(\lambda_1\),
		\item along the direction of \(\bm{v}_2\) by factor \(\lambda_2\),
		\item along the direction of \(\bm{v}_3\) by factor \(\lambda_3\).
	\end{itemize}
\end{itemize}

\subsection{Reflection}
\subsubsection{Reflection in $\mathbb{R}^2$}
\paragraph{Reflection w.r.t. $x$-axis}\,\\
Let \(T:\mathbb{R}^2\to\mathbb{R}^2\) be a \textbf{linear transformation} with \textbf{standard matrix} \(\begin{pmatrix}
	1 & 0 \\ 0 & -1
\end{pmatrix}\)
\[T\left(\begin{pmatrix}
	x \\ y
\end{pmatrix}\right)=\begin{pmatrix}
	x \\ -y
\end{pmatrix}\]
\(T\) is the \textbf{reflection} w.r.t. the \(x\)-axis.

\paragraph{Reflection w.r.t. $y$-axis}\,\\
Let \(T:\mathbb{R}^2\to\mathbb{R}^2\) be a \textbf{linear transformation} with \textbf{standard matrix} \(\begin{pmatrix}
	-1 & 0 \\ 0 & 1
\end{pmatrix}\)
\[T\left(\begin{pmatrix}
	x \\ y
\end{pmatrix}\right)=\begin{pmatrix}
	-x \\ y
\end{pmatrix}\]
\(T\) is the \textbf{reflection} w.r.t. the \(y\)-axis.

\paragraph{Reflection w.r.t. the line $y=x$}\,\\
Let \(T:\mathbb{R}^2\to\mathbb{R}^2\) be a \textbf{linear transformation} with \textbf{standard matrix} \(\begin{pmatrix}
	0 & 1 \\ 1 & 0
\end{pmatrix}\)
\[T\left(\begin{pmatrix}
	x \\ y
\end{pmatrix}\right)=\begin{pmatrix}
	y \\ x
\end{pmatrix}\]
\(T\) is the \textbf{reflection} w.r.t. the line \(y=x\).

\paragraph{Reflection w.r.t. a Line Through the Origin}\,\\
Let \(T:\mathbb{R}^2\to\mathbb{R}^2\) be a \textbf{linear transformation} with \textbf{standard matrix} \(\begin{pmatrix}
	\cos(2\theta) & \sin(2\theta) \\ \sin(2\theta) & -\cos(2\theta)
\end{pmatrix}\)
\(T\) is the \textbf{reflection} w.r.t. the line \(\ell\) passing through the origin where \(\theta\) is the angle between \(\ell\) and the \(x\)-axis.\\ \\
\textbf{Orthogonal Matrices of Determinant $-1$}\\
Every \textbf{orthogonal matrix} of \(\det=-1\) is in this form.
\\ \\
\textbf{Representation with Unit Vector Parallel to Line}
\begin{itemize}
	\item Let \(\bm{n}=(\cos\theta,\sin\theta)^T\) be a unit vector on \(\ell\).
	\item\(\bm{p}\) is the projection of \(\bm{v}\) onto \(\vspan\{\bm{n}\}\)
	\begin{itemize}
		\item\(\bm{p}=(\bm{v}\cdot\bm{n})\bm{n}\)
	\end{itemize}
	\item\(\bm{p}\) is the midpoint of \(\bm{v}\) and \(T(\bm{v})\)
	\begin{itemize}
		\item\(T(\bm{v})=2\bm{p}-\bm{v}=2(\bm{v}\cdot\bm{n})\bm{n}-\bm{v}\)
	\end{itemize}
\end{itemize}
\textbf{Representation with Unit Vector Orthogonal to Line}
\begin{itemize}
	\item Let \(\bm{n}=(\sin\theta,-\cos\theta)^T\) be a unit vector orthogonal to \(\ell\).
	\item\(\bm{p}\) is the projection of \(\bm{v}\) onto \(\vspan\{\bm{n}\}\)
	\begin{itemize}
		\item\(\bm{p}=(\bm{v}\cdot\bm{n})\bm{n}\)
	\end{itemize}
	\item Note that \(T(\bm{v})+2\bm{p}=\bm{v}\)
	\begin{itemize}
		\item\(T(\bm{v})=\bm{v}-2\bm{p}=\bm{v}-2(\bm{v}\cdot\bm{n})\bm{n}\)
	\end{itemize}
\end{itemize}

\subsubsection{Reflections in $\mathbb{R}^3$}
\paragraph{Reflections w.r.t. Planes Formed by Coordinate Axes}\,\\
Let \(T:\mathbb{R}^3\to\mathbb{R}^3\) be a \textbf{linear transformation}.
\begin{itemize}
	\item If the \textbf{standard matrix} is \(\begin{pmatrix}
		1 & 0 & 0 \\ 0 & 1 & 0 \\ 0 & 0 & -1
	\end{pmatrix}\),
	\begin{itemize}
		\item then \(T\) is the reflection w.r.t. the \(xy\)-plane.
	\end{itemize}
	\item If the \textbf{standard matrix} is \(\begin{pmatrix}
		1 & 0 & 0 \\ 0 & -1 & 0 \\ 0 & 0 & 1
	\end{pmatrix}\),
	\begin{itemize}
		\item then \(T\) is the reflection w.r.t. the \(xz\)-plane.
	\end{itemize}
	\item If the \textbf{standard matrix} is \(\begin{pmatrix}
		-1 & 0 & 0 \\ 0 & 1 & 0 \\ 0 & 0 & 1
	\end{pmatrix}\),
	\begin{itemize}
		\item then \(T\) is the reflection w.r.t. the \(yz\)-plane.
	\end{itemize}
\end{itemize}

\paragraph{Reflections w.r.t. Any Plane}\,\\
Let \(T:\mathbb{R}^3\to\mathbb{R}^3\) be the reflection w.r.t. the plane \(ax+by+cz=0\), where \(a,b,c\) not all zero. Then \(\bm{n}=(a,b,c)^T\) and \(\displaystyle\forall\bm{v}\in\mathbb{R}^3,\quad T(\bm{v})=\bm{v}-\left(2\frac{\bm{v}\cdot\bm{n}}{\norm{\bm{n}}^2}\right)\bm{n}\).

\subsection{Rotation}
\subsubsection{Rotation in $\mathbb{R}^2$}
Let \(T:\mathbb{R}^2\to\mathbb{R}^2\) be the \textbf{rotation} about the origin by \(\theta\).
\begin{itemize}
	\item Then \(T\) is a \textbf{linear transformation}.
	\item The \textbf{standard matrix} for \(T\) is \(\begin{pmatrix}
		\cos\theta & -\sin\theta \\ \sin\theta & \cos\theta
	\end{pmatrix}\)
\end{itemize}

\paragraph{Orthogonal Matrices of Determinant $1$}\,\\
Every \textbf{orthogonal matrix} of \(\det=1\) is in this form.

\subsubsection{Rotation in $\mathbb{R}^3$}
\begin{itemize}
	\item Let \(T:\mathbb{R}^3\to\mathbb{R}^3\) be the \textbf{rotation} about the \(z\)-axis anticlockwise by angle \(\theta\).
	\begin{itemize}
		\item The \(z\)-coordinate does not change
		\item It is the rotation about the origin on the plane \(z=z_0\) anticlockwise by \(\theta\).
		\item\textbf{Standard Matrix}\(\begin{pmatrix}
			\cos\theta & -\sin\theta & 0 \\
			\sin\theta & \cos\theta & 0 \\
			0 & 0 & 1 \\
		\end{pmatrix}\)
	\end{itemize}
	\item Let \(T:\mathbb{R}^3\to\mathbb{R}^3\) be the \textbf{rotation} about the \(x\)-axis anticlockwise by angle \(\theta\).
	\begin{itemize}
		\item The \(x\)-coordinate does not change
		\item It is the rotation about the origin on the plane \(x=x_0\) anticlockwise by \(\theta\).
		\item\textbf{Standard Matrix}\(\begin{pmatrix}
			1 & 0 & 0 \\
			0 & \cos\theta & -\sin\theta \\
			0 & \sin\theta & \cos\theta \\
		\end{pmatrix}\)
	\end{itemize}
	\item Let \(T:\mathbb{R}^3\to\mathbb{R}^3\) be the \textbf{rotation} about the \(y\)-axis anticlockwise by angle \(\theta\).
	\begin{itemize}
		\item The \(y\)-coordinate does not change
		\item It is the rotation about the origin on the plane \(y=y_0\) anticlockwise by \(\theta\).
		\item\textbf{Standard Matrix}\(\begin{pmatrix}
			\cos\theta & 0 & -\sin\theta \\
			0 & 1 & 0 \\
			\sin\theta & 0 & \cos\theta \\
		\end{pmatrix}\)
	\end{itemize}
\end{itemize}

\subsection{Reflections and Rotations in $\mathbb{R}^2$}
\subsubsection{Determinant Determines Rotation vs Reflection}
Suppose \textbf{standard matrix} \(\bm{A}\) for \(T:\mathbb{R}^2\to\mathbb{R}^2\) is \textbf{orthogonal}.
\begin{itemize}
	\item If \(\det(\bm{A})=1\), \(T\) represents a \textbf{rotation} about the origin.
	\item If \(\det(\bm{A})=-1\), \(T\) represents the \textbf{reflection} w.r.t. a line passing through the origin.
\end{itemize}

\subsubsection{Reflection about a Line can be Decomposed into Reflection and Rotation}
Since the \textbf{standard matrix} for \textbf{reflection} about a line \(\ell\) passing through the origin where \(\theta\) is the angle between \(\ell\) and the \(x\)-axis is
\[\begin{pmatrix}
	\cos(2\theta) & \sin(2\theta) \\ \sin(2\theta) & -\cos(2\theta)
\end{pmatrix}=\begin{pmatrix}
	\cos(2\theta) & -\sin(2\theta) \\ \sin(2\theta) & \cos(2\theta)
\end{pmatrix}\begin{pmatrix}
	1 & 0 \\ 0 & -1
\end{pmatrix}\]
Reflection w.r.t. \(\ell\) can be decomposed into
\begin{enumerate}
	\item Reflection w.r.t. \(x\)-axis
	\item Rotation about the origin anticlockwise by \(2\theta\).
\end{enumerate}

\subsection{Shears}
\subsubsection{Shears in $\mathbb{R}^2$}
\paragraph{Shear in \(x\)-direction}\,\\
Let \(T:\mathbb{R}^2\to\mathbb{R}^2\) be defined by
\[T\left(\begin{pmatrix}
	x \\ y
\end{pmatrix}\right)=\begin{pmatrix}
	x+ky \\ y
\end{pmatrix}\]
Then \(T\) is a \textbf{shear} in the \(x\)-direction by a factor \(k\).

\paragraph{Shear in \(y\)-direction}\,\\
Let \(T:\mathbb{R}^2\to\mathbb{R}^2\) be defined by
\[T\left(\begin{pmatrix}
	x \\ y
\end{pmatrix}\right)=\begin{pmatrix}
	x \\ kx+y
\end{pmatrix}\]
Then \(T\) is a \textbf{shear} in the \(y\)-direction by a factor \(k\).

\subsubsection{Shears in $\mathbb{R}^3$}
Let \(T:\mathbb{R}^3\to\mathbb{R}^3\) be defined by
\[T\left(\begin{pmatrix}
	x \\ y \\ z
\end{pmatrix}\right)=\begin{pmatrix}
	x+k_1z \\ y+k_2z \\ z
\end{pmatrix}\]
Then \(T\) is a \textbf{shear} in the \(x\)-direction by a factor \(k_1\), and in the \(y\)-direction by factor \(k_2\).
\begin{itemize}
 	\item On \(yz\)-plane \(x=0\), it is a shear in \(y\)-direction by \(k_2\).
 	\item On \(xz\)-plane \(y=0\), it is a shear in \(x\)-direction by \(k_1\).
 	\item On the plane \(z=1\),
 	\begin{itemize}
 		\item\(T\left(\begin{pmatrix}
 			x \\ y \\ 1
 		\end{pmatrix}\right)=\begin{pmatrix}
 			x+k_1 \\ y+k_2 \\ 1
 		\end{pmatrix}\).
 	\end{itemize}
\end{itemize}

\subsection{Translations}
Let \(T:\mathbb{R}^2\to\mathbb{R}^2\) be defined by
\[T\left(\begin{pmatrix}
	x \\ y
\end{pmatrix}\right)=\begin{pmatrix}
	x+a \\ y+b
\end{pmatrix},a,b\in\mathbb{R}\]
\begin{itemize}
	\item\(T\) is a \textbf{translation} by \((a,b)^T\).
	\item\(T\) is \textbf{not} a \textbf{linear transformation} unless \(a=b=0\).
\end{itemize}

\subsection{2D Computer Graphic System}
\subsubsection{Representation of 2D Figures}
\begin{itemize}
	\item In 2D computer graphic, a figure is drawn by connecting points \((a_1,b_1),(a_2,b_2),\ldots,(a_n,b_n)\).
	\item It can be written as an \(2\times n\) matrix:
	\begin{itemize}
		\item\(\bm{M}=\begin{pmatrix}
			a_1 & a_2 & \cdots & a_n \\
			b_1 & b_2 & \cdots & b_n
		\end{pmatrix}\).
	\end{itemize}
\end{itemize}

\subsubsection{Scaling, Reflection, Rotation, Shearing of 2D Figures}
\begin{itemize}
	\item Let \(T\) be a scaling/reflection/rotation/shearing on \(\mathbb{R}^2\)
	\begin{itemize}
		\item Then \(T\) is a \textbf{linear transformation} with \textbf{standard matrix} \(\bm{A}\).
	\end{itemize}
	\item Let \(\bm{M}=\begin{pmatrix}
		\bm{v}_1 & \bm{v}_2 & \cdots & \bm{v}_n
	\end{pmatrix}\) be a 2D figure.
	\item Then the resulting figure by \(T\) is
	\begin{itemize}
		\item\(\begin{pmatrix}
			T(\bm{v}_1) & \cdots & T(\bm{v}_n)
		\end{pmatrix}=\begin{pmatrix}
			\bm{Av}_1 & \cdots & \bm{Av}_n
		\end{pmatrix}=\bm{A}\begin{pmatrix}
			\bm{v}_1 & \cdots & \bm{v}_n
		\end{pmatrix}=\bm{AM}\).
	\end{itemize}
\end{itemize}

\subsubsection{Translating 2D Figure with Homogeneous Coordinate System}
\begin{itemize}
	\item\textbf{Homogeneous coordinate system} is formed by identifying \(\mathbb{R}^2\) with plane \(z=1\) in \(\mathbb{R}^3\): \(\begin{pmatrix}
		a \\ b
	\end{pmatrix}\leftrightarrow\begin{pmatrix}
		a \\ b \\ 1
	\end{pmatrix}\)
	\item A figure \((a_1,b_1),\ldots,(a_n,b_n)\) is identified by \((a_1,b_1,1),\ldots,(a_n,b_n,1)\).
	\item The associated matrix \(\bm{M}=\begin{pmatrix}
		a_1 & \cdots & a_n \\
		b_1 & \cdots & b_n \\
		1 & \cdots & 1
	\end{pmatrix}\)
	\item Suppose we want to do a translation by \((a,b)^T\).
	\item Define shear \(T\left(\begin{pmatrix}
		x \\ y \\ z
	\end{pmatrix}\right)=\begin{pmatrix}
		x+az \\ y+bz \\ z
	\end{pmatrix}\) with \textbf{standard matrix} \(\bm{A}=\begin{pmatrix}
		1 & 0 & a \\
		0 & 1 & b \\
		0 & 0 & 1
	\end{pmatrix}\)
	\item\(\bm{AM}=\begin{pmatrix}
		a_1+a & \cdots & a_n+a \\
		b_1+b & \cdots & b_n+b \\
		1 & \cdots & 1
	\end{pmatrix}\) represent the translation by \((a,b)^T\).
\end{itemize}
\end{document}