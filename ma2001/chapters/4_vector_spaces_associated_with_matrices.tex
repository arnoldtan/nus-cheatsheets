\documentclass[../ma2001_notes.tex]{subfiles}

\begin{document}
\chapter{Vector Spaces Associated With Matrices}

\section{Row Spaces and Column Spaces}
\subsection{Definition of Row Spaces and Column Spaces}
\textbf{Definition.} Let \(\bm{A}=(a_{ij})_{m\times n}\).

\subsubsection{Row Space}
For \(1\leq i\leq m\), let
\begin{align*}
	\bm{r}_i
	&=i\text{th row of } \bm{A} \\
	&=\begin{pmatrix}
		a_{i1} & \cdots & a_{in}
	\end{pmatrix}\in\mathbb{R}^n
\end{align*}
Then the \textbf{row space} of \(\bm{A}=\vspan\{\bm{r}_1,\ldots,\bm{r}_m\}\) is a \textbf{subspace} of \(\mathbb{R}^n\).

\subsubsection{Column Space}
For \(1\leq j\leq n\), let
\begin{align*}
	\bm{c}_j
	&=j\text{th column of } \bm{A} \\
	&=\begin{pmatrix}
		a_{1j} \\ \vdots \\ a_{mj}
	\end{pmatrix}\in\mathbb{R}^m
\end{align*}
Then the \textbf{column space} of \(\bm{A}=\vspan\{\bm{c}_1,\ldots,\bm{c}_n\}\) is a \textbf{subspace} of \(\mathbb{R}^m\).

\subsection{Row Equivalent Matrices And Their Rows}
\subsubsection{Row Equivalence Implies Same Row Space}
\textbf{Theorem.} Suppose \(\bm{A}\) and \(\bm{B}\) are \textbf{row equivalent matrices}. Then \textbf{row space} of \(\bm{A}=\) \textbf{row space} of \(\bm{B}\).

\paragraph{Remark}\,\\
Let \(\bm{R}\) be a \textbf{row-echelon form} of \(\bm{A}\). Then the \textbf{row space} of \(\bm{A}\) = \textbf{row space} of \(\bm{R}\).

\subsubsection{Nonzero Rows of REF are Linearly Independent}
\textbf{Theorem.} Let \(\bm{R}\) be a \textbf{row-echelon form} of a matrix \(\bm{A}\). Then the nonzero rows of \(R\) are \textbf{linearly independent}.

\subsubsection{Finding Basis/Dimension For Row Space}
\textbf{Theorem.} Let \(\bm{R}\) be a \textbf{row-echelon form} of a matrix \(\bm{A}\). Then
\begin{itemize}
	\item the nonzero rows of \(\bm{R}\) form a \textbf{basis} for the \textbf{row space} of \(\bm{A}\), and
	\item\(\dim(\text{row space of }\bm{A})=\) no. of nonzero rows of \(\bm{R}\).
\end{itemize}


\subsection{Row Equivalent Matrices And Their Columns}
\subsubsection{Row Equivalence Preserve Linear Relations Between Columns}
\textbf{Theorem.} Let \(\bm{A}\) and \(\bm{B}\) be \textbf{row equivalent matrices}. Then
\begin{itemize}
	\item There is a linear relation among a given set of columns of \(\bm{A}\Leftrightarrow\) the same linear relation exists among the corresponding set of columns of \(\bm{B}\).
	\item A given set of columns of \(\bm{A}\) is \textbf{linearly independent} \(\Leftrightarrow\) the corresponding set of columns of \(\bm{B}\) is \textbf{linearly independent}.
	\item A given set of columns of \(\bm{A}\) is a \textbf{basis} for the column space of \(\bm{A}\) \(\Leftrightarrow\) the corresponding set of columns of \(\bm{B}\) is a \textbf{basis} for the column space of \(\bm{B}\)
\end{itemize}

\subsubsection{Pivot Columns of REF is a Basis For Column Space}
\textbf{Theorem.} Let \(\bm{R}\) be a \textbf{row-echelon form} of a matrix \(\bm{A}\). Then the \textbf{pivot columns} of \(R\) is a \textbf{basis} for the \textbf{column space} of \(\bm{R}\).

\subsubsection{Finding Basis/Dimension For Column Space}
\textbf{Theorem.} Let \(\bm{R}\) be a \textbf{row-echelon form} of a matrix \(\bm{A}\). The \textbf{pivot columns} of \(\bm{R}\) is a \textbf{basis} for the \textbf{column space} of \(\bm{R}\). Then
\begin{itemize}
	\item the corresponding columns of \(\bm{A}\) is a \textbf{basis} for the column space of \(\bm{A}\), and
	\item\(\dim(\text{column space of }\bm{A})=\) no. of \textbf{pivot columns} of \(\bm{R}\).
\end{itemize}

\subsection{Finding Basis for Vector Space Spanned By a Set of Vectors}
To find a \textbf{basis} for a \textbf{vector space} \(\bm{V}=\vspan\{\bm{v}_1,\ldots,\bm{v}_m\}\), choose either one of the methods below:

\subsubsection{Method 1: From New Vectors Using Row Space}
\begin{enumerate}
	\item Let each \(\bm{v}_1,\ldots,\bm{v}_m\) be a \textbf{row vector}
	\item Let matrix \(\bm{A}=\begin{pmatrix}
		\bm{v}_1 \\ \vdots \\ \bm{v}_m
	\end{pmatrix}\)
	\item The problem is now equivalent to finding the \textbf{basis} for the \textbf{row space} of \(\bm{A}\) (refer to section 1.1.2.3).
\end{enumerate}

\subsubsection{Method 2: Select From Original Vectors Using Column Space}
\begin{enumerate}
	\item Let each \(\bm{v}_1,\ldots,\bm{v}_m\) be a \textbf{column vector}
	\item Let matrix \(\bm{A}=\begin{pmatrix}
		\bm{v}_1 & \cdots & \bm{v}_m
	\end{pmatrix}\)
	\item The problem is now equivalent to finding the \textbf{basis} for the \textbf{column space} of \(\bm{A}\) (refer to section 1.1.3.3).
\end{enumerate}

\subsection{Consistency of Linear Systems And Column Space of Coefficient Matrix}
\textbf{Theorem.} Let \(\bm{A}\) be a \(m\times n\) \textbf{matrix}. The \textbf{linear system} \(\bm{Ax}=\bm{b}\) is \textbf{consistent} \(\Leftrightarrow\bm{b}\in\{\bm{Av}\mid\bm{v}\in\mathbb{R}^n\}=\) column space of \(\bm{A}\).

\section{Rank}
\subsection{The Dimension of Row Space And Column Space Are The Same}
\textbf{Theorem.} Let \(\bm{A}\) be a \textbf{matrix}. Then
\[\dim(\text{row space of }\bm{A})=\dim(\text{column space of }\bm{A})\]

\subsection{Definition of Rank}
\textbf{Definition.} Let \(\bm{A}\) be a \textbf{matrix}. The \textbf{rank} of \(\bm{A}\), denoted by \(\rank(\bm{A})\) is
\[\rank(\bm{A})=\dim(\text{row space of }\bm{A})=\dim(\text{column space of }\bm{A})\]

\subsection{Properties of Rank}
\textbf{Theorem.} Let \(\bm{A}\) be a \(m\times n\) \textbf{matrix}. Then
\begin{itemize}
	\item\(\rank(\bm{A})=\rank(\bm{A}^T)\)
	\item\(\rank(\bm{A})=0\Leftrightarrow\bm{A}=\bm{0}\)
	\item\(\rank(\bm{A})\leq m\) and \(\rank(\bm{A})\leq n\)
	\begin{itemize}
		\item\(\rank(\bm{A})\leq\min\{m,n\}\)
		\item\(\bm{A}\) is \textbf{full rank} if \(\rank(\bm{A})=\min\{m,n\}\)
	\end{itemize}
	\item A \textbf{square matrix} \(\bm{A}\) is \textbf{full rank} \(\Leftrightarrow\bm{A}\) is \textbf{invertible}.
\end{itemize}

\subsection{Rank \& Consistency of Linear System}
\textbf{Theorem.} Let \(\bm{Ax}=\bm{b}\) be a \textbf{linear system}. Let \(\left(\begin{array}{c|c}
	\bm{R} & \bm{b}'
\end{array}\right)\) be the \textbf{row-echelon form} of \(\left(\begin{array}{c|c}
	\bm{A} & \bm{b}
\end{array}\right)\). Then
\begin{align*}
	\bm{Ax}=\bm{b}\text{ is }\textbf{consistent}
	&\Leftrightarrow\rank(\bm{A})=\rank\left(\begin{array}{c|c}
		\bm{A} & \bm{b}
	\end{array}\right) \\
	&\Leftrightarrow\rank(\bm{R})=\rank\left(\begin{array}{c|c}
	\bm{R} & \bm{b}'
	\end{array}\right)
\end{align*}

\subsubsection{Remark}
In general, \(\rank(\bm{A})\leq\rank\left(\begin{array}{c|c}
	\bm{A} & \bm{b}
\end{array}\right)\leq\rank(\bm{A})+1\)

\subsection{Row/Column Spaces and Matrix Multiplication}
\textbf{Theorem.} Let \(\bm{A}\) be a \(m\times n\) \textbf{matrix} and \(\bm{B}\) be a \(n\times p\) \textbf{matrix}. Then
\begin{itemize}
	\item column space of \(\bm{AB}\subseteq\) column space of \(\bm{A}\);
	\item row space of \(\bm{AB}\subseteq\) row space of \(\bm{B}\).
\end{itemize}

\subsubsection{Ranks and Matrix Multiplication}
In particular,
\begin{itemize}
	\item\(\rank(\bm{AB})\leq\rank(\bm{A})\);
	\item\(\rank(\bm{AB})\leq\rank(\bm{B})\).
\end{itemize}
That is, \(\rank(\bm{AB})\leq\min\{\rank(\bm{A}),\rank(\bm{B})\}\).
\begin{itemize}
	\item\(\bm{A}\) is \textbf{invertible} \(\implies\rank(\bm{AB})=\rank(\bm{B})\)
	\item\(\bm{B}\) is \textbf{invertible} \(\implies\rank(\bm{AB})=\rank(\bm{A})\)
\end{itemize}

\section{Nullspaces and Nullities}
\subsection{Definition of Nullspace}
\textbf{Definition.} Let \(\bm{A}\) be a \(m\times n\) \textbf{matrix}. The \textbf{nullspace} of \(\bm{A}\) is the \textbf{solution space} of \(\bm{Ax}=\bm{0}\), which is
\[\{\bm{v}\in\mathbb{R}^n\mid\bm{Av}=\bm{0}\}.\]

\subsection{Definition of Nullity}
\textbf{Definition.} The \textbf{dimension} of the \textbf{nullspace} of a \textbf{matrix} \(\bm{A}\) is the \textbf{nullity} of \(\bm{A}\), denoted by \(\nullity(\bm{A})\).

\subsection{Finding Nullity of a Matrix}
Let \(\bm{R}\) be a \textbf{row-echelon form} of \(\bm{A}\). Then
\begin{itemize}
 	\item\(\bm{Ax}=\bm{0}\Leftrightarrow\bm{Rx}=\bm{0}\)
 	\item\textbf{nullspace} of \(\bm{A}=\) \textbf{nullspace} of \(\bm{R}\)
 	\item\(\nullity(\bm{A})=\nullity(\bm{R})=\) no. of \textbf{non-pivot columns} of \(\bm{R}\).
\end{itemize}

\subsection{Dimension Theorem}
\textbf{Theorem.} Let \(\bm{A}\) be a \(m\times n\) \textbf{matrix}. Then
\[\rank(\bm{A})+\nullity(\bm{A})=n\]

\subsection{Solution to Inhomogeneous Linear System and Nullspace}
\textbf{Theorem.} Suppose \(\bm{Ax}=\bm{b}\) has a \textbf{solution} \(\bm{v}\). Let \(W=\) \textbf{nullspace} of \(\bm{A}\). Then the \textbf{solution set} of \(\bm{Ax}=\bm{b}\) is
\[\bm{v}+W=\{\bm{v}+\bm{w}\mid\bm{w}\in W\}\]
Hence
\[(\text{A general solution of }\bm{Ax}=\bm{b})=(\text{A particular solution of }\bm{Ax}=\bm{b})+(\text{A general solution of }\bm{Ax}=\bm{0})\]

\subsubsection{Condition for Linear System to have Unique Solution}
\textbf{Theorem.} Suppose that a \textbf{linear system} \(\bm{Ax}=\bm{b}\) is \textbf{consistent}. Then
\begin{align*}
	\bm{Ax}=\bm{b}\text{ has a unique }\textbf{solution}
	&\Leftrightarrow\bm{Ax}=\bm{0}\text{ has only the }\textbf{trivial solution} \\
	&\Leftrightarrow\textbf{nullspace}\text{ of }\bm{A}\text{ is }\{\bm{0}\} \\
	&\Leftrightarrow\nullity(\bm{A})=0 \\
	&\Leftrightarrow\rank(\bm{A})=\text{no. of columns of }\bm{A}
\end{align*}

\subsubsection{Condition for the Solution Set of a Linear System to be a Vector Space}
\textbf{Theorem.} The \textbf{solution set} of a \textbf{linear system} \(\bm{Ax}=\bm{b}\) is a \textbf{vector space} \(\Leftrightarrow\bm{b}=\bm{0}\).
\end{document}