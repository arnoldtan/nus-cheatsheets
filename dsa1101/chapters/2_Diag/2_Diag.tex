\documentclass[../../dsa1101_notes.Rtex]{subfiles}\usepackage[]{graphicx}\usepackage[]{color}
% maxwidth is the original width if it is less than linewidth
% otherwise use linewidth (to make sure the graphics do not exceed the margin)
\makeatletter
\def\maxwidth{ %
  \ifdim\Gin@nat@width>\linewidth
    \linewidth
  \else
    \Gin@nat@width
  \fi
}
\makeatother

\definecolor{fgcolor}{rgb}{0.345, 0.345, 0.345}
\newcommand{\hlnum}[1]{\textcolor[rgb]{0.686,0.059,0.569}{#1}}%
\newcommand{\hlstr}[1]{\textcolor[rgb]{0.192,0.494,0.8}{#1}}%
\newcommand{\hlcom}[1]{\textcolor[rgb]{0.678,0.584,0.686}{\textit{#1}}}%
\newcommand{\hlopt}[1]{\textcolor[rgb]{0,0,0}{#1}}%
\newcommand{\hlstd}[1]{\textcolor[rgb]{0.345,0.345,0.345}{#1}}%
\newcommand{\hlkwa}[1]{\textcolor[rgb]{0.161,0.373,0.58}{\textbf{#1}}}%
\newcommand{\hlkwb}[1]{\textcolor[rgb]{0.69,0.353,0.396}{#1}}%
\newcommand{\hlkwc}[1]{\textcolor[rgb]{0.333,0.667,0.333}{#1}}%
\newcommand{\hlkwd}[1]{\textcolor[rgb]{0.737,0.353,0.396}{\textbf{#1}}}%
\let\hlipl\hlkwb

\usepackage{framed}
\makeatletter
\newenvironment{kframe}{%
 \def\at@end@of@kframe{}%
 \ifinner\ifhmode%
  \def\at@end@of@kframe{\end{minipage}}%
  \begin{minipage}{\columnwidth}%
 \fi\fi%
 \def\FrameCommand##1{\hskip\@totalleftmargin \hskip-\fboxsep
 \colorbox{shadecolor}{##1}\hskip-\fboxsep
     % There is no \\@totalrightmargin, so:
     \hskip-\linewidth \hskip-\@totalleftmargin \hskip\columnwidth}%
 \MakeFramed {\advance\hsize-\width
   \@totalleftmargin\z@ \linewidth\hsize
   \@setminipage}}%
 {\par\unskip\endMakeFramed%
 \at@end@of@kframe}
\makeatother

\definecolor{shadecolor}{rgb}{.97, .97, .97}
\definecolor{messagecolor}{rgb}{0, 0, 0}
\definecolor{warningcolor}{rgb}{1, 0, 1}
\definecolor{errorcolor}{rgb}{1, 0, 0}
\newenvironment{knitrout}{}{} % an empty environment to be redefined in TeX

\usepackage{alltt}
\IfFileExists{upquote.sty}{\usepackage{upquote}}{}
\begin{document}
\section{Diagnostics of Classifiers}
\subsection{Confusion Matrix}
\begin{center}
\begin{tabular}{|c|c|c|c|}
\cline{3-4}
\multicolumn{2}{c}{} & \multicolumn{2}{|c|}{Predicted Class}\\
\cline{3-4}
\multicolumn{2}{c}{} & \multicolumn{1}{|c|}{Positive} & Negative\\
\hline
\multirow{2}{3em}{Actual Class} & \multicolumn{1}{c}{Positive} & \multicolumn{1}{|c|}{True Positive (TP)} & False Negative (FN)\\
\cline{2-4}
& \multicolumn{1}{c}{Negative} & \multicolumn{1}{|c|}{False Positive (FP)} & True Negative (TN)\\
\hline
\end{tabular}
\end{center}

\subsection{Accuracy}
\[\text{Accuracy} = \frac{TP + TN}{TP + TN + FP + FN}\]

\subsection{True Positive Rate}
\[\text{TPR} = \frac{TP}{TP + FN}\]

\subsection{False Positive Rate / Type I Error Rate}
\[\text{FPR} = \frac{FP}{FP + TN}\]

\subsection{False Negative Rate / Type II Error Rate}
\[\text{FNR} = \frac{FN}{TP + FN}\]

\subsection{True Negative Rate}
\[\text{TNR} = \frac{TN}{TN+FP}\]

\subsection{Precision}
\[\text{Precision} = \frac{TP}{TP + FP}\]

\subsubsection{Remarks}
\begin{enumerate}
  \item Precision is useful when costly actions will be followed up on the data predicted to be positive,
  \item because precision gives the proportion of actual positives among those predicted to be positive
  \item For example, if an insurance company wants to predict potential customers interested in purchasing insurance, and the cost to try to sell an insurance to a potential customer is non-trivial (e.g. insurance agent has to house visit the customer).
\end{enumerate}

\subsection{N-Fold Cross Validation}
\subsubsection{Algorithm}
\begin{enumerate}
	\item The entire dataset is randomly split into \(N\) datasets of approximately equal size.
	\item \(N-1\) of these datasets are treated as the training dataset, while the remaining one is the test dataset. A measure of the model error is obtained.
	\item This process is repeated across the various combinations of N datasets taken \(N-1\) at a time.
	\item The observed N models errors are averaged across the N folds
\end{enumerate}

\subsection{ROC Curve (TPR vs FPR Trade-off)}
\begin{enumerate}
    \item Graph of True Positive Rate (TPR) against False Positive Rate (FPR)
    \item As TPR increases, FPR tend to increase as well
        \begin{enumerate}
            \item Increasing TPR may be a double-edged sword
        \end{enumerate}
    \item TPR = FPR = 0 means binary classifier classifies everything as negative
    \item TPR = FPR = 1 means binary classifier classifies everything as positive
\end{enumerate}

\subsection{Bias-Variance tradeoff}
\begin{enumerate}
    \item \(\text{error} = \text{bias}^2 + \text{variance} + \text{irreducible error}\)
    \item As variance increases, bias decreases, and vice versa
\end{enumerate}

\subsection{Calculation Intensive Exam Question \& Solution}
\textbf{Midterm Q6.} Consider the following confusion matrix for a classifier
\begin{center}
\begin{tabular}{|c|c|c|c|}
\cline{3-4}
\multicolumn{2}{c}{} & \multicolumn{2}{|c|}{Predicted Class}\\
\cline{3-4}
\multicolumn{2}{c}{} & \multicolumn{1}{|c|}{Positive} & Negative\\
\hline
\multirow{2}{3em}{Actual Class} & \multicolumn{1}{c}{Positive} & \multicolumn{1}{|c|}{20} & 75\\
\cline{2-4}
& \multicolumn{1}{c}{Negative} & \multicolumn{1}{|c|}{140} & 55\\
\hline
\end{tabular}
\end{center}
The false negative rate (FNR) of the classifier is \rule{1cm}{0.15mm} (round to 3 decimal places).\\
\textbf{Solution}
\begin{enumerate}
  \item Copy paste the following code:
\begin{knitrout}
\definecolor{shadecolor}{rgb}{0.969, 0.969, 0.969}\color{fgcolor}\begin{kframe}
\begin{alltt}
\hlstd{gcmfv} \hlkwb{<-} \hlkwa{function}\hlstd{(}\hlkwc{tp}\hlstd{,} \hlkwc{fn}\hlstd{,} \hlkwc{fp}\hlstd{,} \hlkwc{tn}\hlstd{) \{}
  \hlcom{# generates confusion matrix from values}
  \hlkwd{return} \hlstd{(}\hlkwd{matrix}\hlstd{(}\hlkwd{c}\hlstd{(tp, fn, fp, tn),} \hlkwc{nrow} \hlstd{=} \hlnum{2}\hlstd{,} \hlkwc{ncol} \hlstd{=} \hlnum{2}\hlstd{,} \hlkwc{byrow} \hlstd{=} \hlnum{TRUE}\hlstd{))}
\hlstd{\}}

\hlstd{tp} \hlkwb{<-} \hlkwa{function}\hlstd{(}\hlkwc{m}\hlstd{) \{}
  \hlcom{# true positive from confusion matrix m}
  \hlkwd{return} \hlstd{(m[}\hlnum{1}\hlstd{,} \hlnum{1}\hlstd{])}
\hlstd{\}}

\hlstd{fn} \hlkwb{<-} \hlkwa{function}\hlstd{(}\hlkwc{m}\hlstd{) \{}
  \hlcom{# false negative from confusion matrix m}
  \hlkwd{return} \hlstd{(m[}\hlnum{1}\hlstd{,} \hlnum{2}\hlstd{])}
\hlstd{\}}

\hlstd{fp} \hlkwb{<-} \hlkwa{function}\hlstd{(}\hlkwc{m}\hlstd{) \{}
  \hlcom{# false positive from confusion matrix m}
  \hlkwd{return} \hlstd{(m[}\hlnum{2}\hlstd{,} \hlnum{1}\hlstd{])}
\hlstd{\}}

\hlstd{tn} \hlkwb{<-} \hlkwa{function}\hlstd{(}\hlkwc{m}\hlstd{) \{}
  \hlcom{# true negative from confusion matrix m}
  \hlkwd{return} \hlstd{(m[}\hlnum{2}\hlstd{,} \hlnum{2}\hlstd{])}
\hlstd{\}}

\hlstd{accuracy} \hlkwb{<-} \hlkwa{function}\hlstd{(}\hlkwc{m}\hlstd{) \{}
  \hlkwd{return} \hlstd{((}\hlkwd{tp}\hlstd{(m)}\hlopt{+}\hlkwd{tn}\hlstd{(m))}\hlopt{/}\hlstd{(}\hlkwd{tp}\hlstd{(m)}\hlopt{+}\hlkwd{tn}\hlstd{(m)}\hlopt{+}\hlkwd{fp}\hlstd{(m)}\hlopt{+}\hlkwd{fn}\hlstd{(m)))}
\hlstd{\}}

\hlstd{tpr} \hlkwb{<-} \hlkwa{function}\hlstd{(}\hlkwc{m}\hlstd{) \{}
  \hlkwd{return} \hlstd{(}\hlkwd{tp}\hlstd{(m)}\hlopt{/}\hlstd{(}\hlkwd{tp}\hlstd{(m)}\hlopt{+}\hlkwd{fn}\hlstd{(m)))}
\hlstd{\}}

\hlstd{fpr} \hlkwb{<-} \hlkwa{function}\hlstd{(}\hlkwc{m}\hlstd{) \{}
  \hlkwd{return} \hlstd{(}\hlkwd{fp}\hlstd{(m)}\hlopt{/}\hlstd{(}\hlkwd{fp}\hlstd{(m)}\hlopt{+}\hlkwd{tn}\hlstd{(m)))}
\hlstd{\}}

\hlstd{fnr} \hlkwb{<-} \hlkwa{function}\hlstd{(}\hlkwc{m}\hlstd{) \{}
  \hlkwd{return} \hlstd{(}\hlkwd{fn}\hlstd{(m)}\hlopt{/}\hlstd{(}\hlkwd{fn}\hlstd{(m)}\hlopt{+}\hlkwd{tp}\hlstd{(m)))}
\hlstd{\}}

\hlstd{tnr} \hlkwb{<-} \hlkwa{function}\hlstd{(}\hlkwc{m}\hlstd{) \{}
  \hlkwd{return} \hlstd{(}\hlkwd{tn}\hlstd{(m)}\hlopt{/}\hlstd{(}\hlkwd{tn}\hlstd{(m)}\hlopt{+}\hlkwd{fp}\hlstd{(m)))}
\hlstd{\}}

\hlstd{precision} \hlkwb{<-} \hlkwa{function}\hlstd{(}\hlkwc{m}\hlstd{) \{}
  \hlkwd{return} \hlstd{(}\hlkwd{tp}\hlstd{(m)}\hlopt{/}\hlstd{(}\hlkwd{tp}\hlstd{(m)}\hlopt{+}\hlkwd{fp}\hlstd{(m)))}
\hlstd{\}}
\end{alltt}
\end{kframe}
\end{knitrout}
\item Create Confusion Matrix
\begin{knitrout}
\definecolor{shadecolor}{rgb}{0.969, 0.969, 0.969}\color{fgcolor}\begin{kframe}
\begin{alltt}
\hlstd{confusion.matrix} \hlkwb{<-} \hlkwd{gcmfv}\hlstd{(}\hlkwc{tp}\hlstd{=}\hlnum{20}\hlstd{,} \hlkwc{fn}\hlstd{=}\hlnum{75}\hlstd{,} \hlkwc{fp}\hlstd{=}\hlnum{140}\hlstd{,} \hlkwc{tn}\hlstd{=}\hlnum{55}\hlstd{)}
\end{alltt}
\end{kframe}
\end{knitrout}
\item Get the metric you need
\begin{knitrout}
\definecolor{shadecolor}{rgb}{0.969, 0.969, 0.969}\color{fgcolor}\begin{kframe}
\begin{alltt}
\hlkwd{fnr}\hlstd{(confusion.matrix)}
\end{alltt}
\begin{verbatim}
## [1] 0.7894737
\end{verbatim}
\end{kframe}
\end{knitrout}
\end{enumerate}
\end{document}
