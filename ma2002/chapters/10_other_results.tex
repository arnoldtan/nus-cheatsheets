\documentclass[../ma2002_notes.tex]{subfiles}
\begin{document}
\section{Other Results}
\subsection{Geometry}
\subsubsection{Slope of Perpendicular Lines}
Given a pair of perpendicular lines \(\ell_1\) and \(\ell_2\) with slopes \(m_1\) and \(m_2\) respectively, then
\[m_1\cdot m_2=-1\]

\subsubsection{Circle}
The equation of a circle with radius \(r\) and centre \((a,b)\) is
\[(x-a)^2+(y-b)^2=r^2\]

\subsubsection{Ellipse}
The equation of an ellipse with width \(2a\), height \(2b\), and center \((x_0,y_0)\) is
\[\frac{(x-x_0)^2}{a^2}+\frac{(y-y_0)^2}{b^2}=1\]

\subsubsection{Hyperbola}
The equation of a hyperbola is
\[\frac{x^2}{a^2}-\frac{y^2}{b^2}=1\]

\subsubsection{Astroid}
The equation of an astroid with radius of the fixed circle \(a\) is
\[x^{2/3}+y^{2/3}=a^{2/3}\]

\subsubsection{Sphere}
\paragraph{Volume}
The volume \(V\) of a sphere with radius \(r\) is
\[V=\frac{4}{3}\pi r^3\]

\paragraph{Surface Area}
The surface area \(A\) of a sphere with radius \(r\) is
\[A=4\pi r^2\]

\subsubsection{Right Cylinder}
\paragraph{Volume}
The volumne \(V\) of a right cylinder with base area \(A\) and height \(H\) is
\[V=AH\]

\subsubsection{Cone}
\paragraph{Volume}
The volume \(V\) of a cone with base area \(A\) and height \(H\) is
\[V=\frac{1}{3}AH\]

\subsubsection{Right Circular Cone}
\paragraph{Volume}
The volume \(V\) of a right circular cone with radius \(r\) and height \(h\) is
\[V=\frac{1}{3}\pi r^2h\]

\paragraph{Surface Area}
A right circular cone with radius \(r\) and height \(h\) has slant height \(l=\sqrt{r^2+h^2}\). Its surface area \(A\) is
\[A=\pi r^2+\pi rl\]

\subsection{Triangle Inequality}
For any \(a,\,b\in\mathbb{R}\),
\[\abs{a}-\abs{b}\leq\abs{a+b}\leq\abs{a}+\abs{b}\]

\subsection{Trigonometric Identities}
\subsubsection{Sum and Difference Formula}
\[\sin(A\pm B) = \sin A \cos B \pm \cos A \sin B\]
\[\cos(A\pm B) = \cos A \cos B \mp \sin A \sin B \]

\subsubsection{Double-Angle Formulae}
\begin{align*}
	&\sin(2\theta)=2\sin\theta\cos\theta=\frac{2\tan\theta}{1+\tan^2\theta}\\
	&\cos(2\theta)=\cos^2\theta-\sin^2\theta=2\cos^2\theta-1=1-2\sin^2\theta=\frac{1-\tan^2\theta}{1+\tan^2\theta}\\
	&\tan(2\theta)=\frac{2\tan\theta}{1-\tan^2\theta}\\
	&\cot(2\theta)=\frac{\cot^2\theta-1}{2\cot\theta}\\
	&\sec(2\theta)=\frac{\sec^2\theta}{2-\sec^2\theta}\\
	&\csc(2\theta)=\frac{\sec\theta\csc\theta}{2}
\end{align*}

\subsubsection{Others}
\(\frac{\sin x}{x} < 1\) and \(\abs{\sin x} \leq \abs{x}\) \\ \\
For \(x \in (0, \frac{\pi}{2})\):
\[x < \tan x\]

\subsection{Algebraic Identities}
\subsubsection{Difference of Cubes Formula}
\[b^3-c^3=(b-c)(b^2+bc+c^2)\]

\subsubsection{Difference of Nth Power Formula}
\[b^n-c^n=(b-c)(b^{n-1}+b^{n-2}c+b^{n-3}c^2+\ldots+bc^{n-2}+c^{n-1})\]

\subsection{Binomial Theorem}
For \(n\in\mathbb{Z},n\geq0\),
\[(x+y)^n=\sum_{k=0}^{n}\binom{n}{k}x^{n-k}y^k\]
where \(\displaystyle\binom{n}{k}=\frac{n!}{k!(n-k)!}\)
\end{document}