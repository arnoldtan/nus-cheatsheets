\documentclass[../ma2002_notes.tex]{subfiles}
\begin{document}
\section{Derivatives}
\subsection{Definition of Derivative at a point}
The \textbf{derivative} of a function \(f\) at \(a\) is the limit
\[f'(a)=\lim_{h\to0}\frac{f(a+h)-f(a)}{h}=\lim_{x\to a}\frac{f(x)-f(a)}{x-a}\]
If \(f'(a)\) exists, then \(f\) is \textbf{differentiable} at \(a\).

\subsection{Derivative as a Function}
The \textbf{derivative} of a function \(f\) is the function
\[f'(x)=\lim_{h\to0}\frac{f(x+h)-f(x)}{h}=\lim_{z\to x}\frac{f(z)-f(x)}{z-x}\text{, if the limit exists}\]

\subsection{Differentiability on an open interval}
A function \(f\) is differentiable on an open interval \(I\) if
\[f\text{ is differentiable at every point in }I\]

\subsection{Differentiability implies Continuity}
If a function \(f\) is differentiable at \(a\), then \(f\) is continuous at \(a\)

\subsection{Differentiation Formulas}
Suppose \(f\) and \(g\) are differentiable at \(x\) and \(c\in\mathbb{R}\)
\subsubsection{Constant Functions}
\[\frac{d}{dx}(c)=0\]

\subsubsection{Constant Multiple}
\[(cf)'(x)=cf'(x)\]

\subsubsection{Sum}
\[(f+g)'(x)=f'(x)+g'(x)\]

\subsubsection{Difference}
\[(f-g)'(x)=f'(x)-g'(x)\]

\subsubsection{Product}
\[(fg)'(x)=f'(x)g(x)+f(x)g'(x)\]

\subsubsection{Quotient}
\[\left(\frac{f}{g}\right)'(x)=\frac{f'(x)g(x)-f(x)g'(x)}{[g(x)]^2}\]

\subsubsection{Integer Power}
\[\forall n\in\mathbb{Z}\quad\frac{d}{dx}x^n=nx^{n-1}\]

\subsection{Differentiable Functions}
\subsubsection{Polynomials}
Every polynomial is differentiable on \(\mathbb{R}\)

\subsubsection{Rational Functions}
Every rational function is differentiable on its domain

\subsubsection{Trigonometric Functions}
Trigonometric Functions are differentiable on their domain
\begin{align*}
	\frac{d}{dx}\sin x&=\cos x\\
	\frac{d}{dx}\cos x&=-\sin x\\
	\frac{d}{dx}\tan x&=\sec^2x\\
	\frac{d}{dx}\cot x&=-\csc^2x\\
	\frac{d}{dx}\sec x&=\sec x\tan x\\
	\frac{d}{dx}\csc x&=-\csc x\cot x
\end{align*}

\subsection{Chain Rule}
If \(f\) is differentiable at \(x\) and \(g\) is differentiable at \(f(x)\), then
\[(g\circ f)'(x)=g'(f(x))f'(x)\]

\subsection{Implicit Differentiation}
\subsubsection{Implicit Function}
Let \(f(x,y)=0\) be an equation in \(x\) and \(y\). If \(y\) can be expressed in \(x\) near a point on \(f(x,y)=0\), then
\[y\text{ is an \textbf{implicit function} of }x\text{ near the point}\]

\subsubsection{Implicit Differentiation}
Furthermore, if \(y\) is an implicit function of \(x\) such that \(\frac{dy}{dx}\) exists, then \(\frac{dy}{dx}\) can be evaluated by implicit differentiation as follows:
\begin{enumerate}
	\item Differentiate \(f(x,y)=0\) w.r.t. \(x\), regarding \(y\) as a differentiable function in \(x\)
	\item Solve for \(\frac{dy}{dx}\) in terms of \(x\) and \(y\)
\end{enumerate}

\subsection{Higher Derivatives}
Let \(f\) be a function
\begin{enumerate}
	\item The \textbf{zeroth derivative} of \(f\) is \(f=f^{(0)}\)
	\item For \(n\in\mathbb{N}\), the \(\pmb{n^{th}}\textbf{ derivative}\) of \(f\) is \(f^{(n)}=(f^{(n-1)})'\)
	\begin{enumerate}
		\item \(f\) is \(\pmb{n}\)\textbf{ times differentiable} if \(f^{(n)}\) exists
	\end{enumerate}
	\item Let \(y=f(x)\). \(f^{(n)}(x)=\frac{d^ny}{dx^n}\)
	\begin{enumerate}
		\item This is the \(n^{\text{th}}\) derivative of \(y\) w.r.t. \(x\)
	\end{enumerate}
\end{enumerate}
\end{document}