\documentclass[../ma2002_notes.tex]{subfiles}
\begin{document}
\section{Applications of Derivatives}
\subsection{Extreme Values}
Let \(f\) be a function with domain \(D\)
\subsubsection{Absolute Maximum}
\(f\) has an \textbf{absolute/global maximum} value at \(c\in D\) if
\[\forall x\in D\quad f(c)\geq f(x)\]
\subsubsection{Absolute Minimum}
\(f\) has an \textbf{absolute/global minimum} value at \(c\in D\) if
\[\forall x\in D\quad f(c)\leq f(x)\]

\subsection{Local Extreme Values}
Let \(f\) be a function with domain \(D\)
\subsubsection{Local Maximum}
\(f\) has an \textbf{relative/local maximum} value at \(c\in D\) if
\[\text{For all }x\text{ in an open interval containing }c,\,f(c)\geq f(x)\]
\subsubsection{Local Minimum}
\(f\) has an \textbf{relative/local minimum} value at \(c\in D\) if
\[\text{For all }x\text{ in an open interval containing }c,\,f(c)\leq f(x)\]

\subsection{Extreme Value Theorem}
If \(f\) is continuous on a finite closed interval \([a,b]\). Then
\[f\text{ attains the extreme values on }[a,b]\]
Precisely, there exist \(c,d\in[a,b]\) such that
\[\forall x\in[a,b]\quad f(c)\leq f(x)\leq f(d)\]

\subsection{Extreme Value Problem}
If \(f\) is continuous on a finite closed interval \([a,b]\)
\begin{enumerate}
	\item Evaluate the values of \(f\) at endpoints: \(f(a)\) and \(f(b)\)
	\item Find local extreme values of \(f\) on \((a,b)\)
	\item Compare the values obtained in Steps 1 and 2:
	\begin{enumerate}
		\item The largest value is the absolute maxmimum value
		\item The smallest value is the absolute minimum value
	\end{enumerate}
\end{enumerate}

\subsection{Fermat's Theorem}
Let \(f\) be a function such that
\begin{enumerate}
	\item \(f\) has a local extreme value at \(c\), and
	\item \(f\) is differentiable at \(c\)
\end{enumerate}
Then \(f'(c)=0\)

\subsubsection{Remarks}
Equivalently, if \(f\) has a local extreme value at \(c\), then either
\begin{enumerate}
	\item \(f\) is not differentiable at \(c\), or
	\item \(f'(c)=0\)
\end{enumerate}

\subsection{Critical Point}
Let \(c\) be an interior point of the domain of \(f\). \(c\) is a critical point of \(f\) if either
\begin{enumerate}
	\item \(f'(c)\) does not exist, or
	\item \(f'(c)=0\)
\end{enumerate}

\subsection{Stationary Point}
\(c\) is a stationary point of \(f\) if \(f'(c)=0\)

\subsection{Closed Interval Method}
Let \(f\) be continuous on \([a,b]\)
\begin{enumerate}
	\item Evaluate the values of \(f\) at endpoints: \(f(a)\) and \(f(b)\)
	\item Evaluate the values of \(f\) at critical points on \((a,b)\)
	\item Compare the values obtained in Steps 1 and 2:
	\begin{enumerate}
		\item The largest value is the absolute maxmimum value
		\item The smallest value is the absolute minimum value
	\end{enumerate}
\end{enumerate}

\subsection{Rolle's Theorem}
If \(f\) is a function such that
\begin{enumerate}
	\item it is continuous on \([a,b]\)
	\item it is differentiable on \((a,b)\), and
	\item \(f(a)=f(b)\)
\end{enumerate}
Then there exists a number \(c\in(a,b)\) such that \(f'(c)=0\)

\subsection{Mean Value Theorem}
If \(f\) is a function such that
\begin{enumerate}
	\item it is continuous on \([a,b]\)
	\item it is differentiable on \((a,b)\)
\end{enumerate}
Then there exists a number \(c\in(a,b)\) such that
\[f'(c)=\frac{f(b)-f(a)}{b-a}\]

\subsection{Derivative Zero Implies Constant Function}
If \(f\) is a function such that
\begin{enumerate}
	\item it is continuous on an interval \(I\)
	\item it is differentiable on \(I'=\) the interior of \(I\), and
	\item \(\forall x\in I'\quad f'(x)=0\)
\end{enumerate}
Then there exists a constant \(C\in\mathbb{R}\) such that
\[\forall x\in I\quad f(x)=C\]

\subsection{Functions with Same Derivative Differ By A Constant}
If \(f\) and \(g\) are functions such that
\begin{enumerate}
	\item they are continuous on an interval \(I\)
	\item they are differentiable on \(I'=\) the interior of \(I\), and
	\item \(\forall x\in I'\quad f'(x)=g'(x)\)
\end{enumerate}
Then there exists a constant \(C\in\mathbb{R}\) such that
\[\forall x\in I\quad f(x)=g(x)+C\]

\subsection{Increasing Function}
A function \(f\) is increasing on a set \(I\) if
\[\text{For any } a,b\in I\quad a<b\implies f(a)<f(b)\]

\subsection{Decreasing Function}
A function \(f\) is decreasing on a set \(I\) if
\[\text{For any } a,b\in I\quad a<b\implies f(a)>f(b)\]

\subsection{Increasing Test}
If \(f\) is a function such that
\begin{enumerate}
	\item it is continuous on an interval \(I\)
	\item it is differentiable on \(I'=\) the interior of \(I\), and
	\item \(\forall x\in I'\quad f'(x)>0\)
\end{enumerate}
Then \(f\) is increasing on \(I\)

\subsection{Decreasing Test}
If \(f\) is a function such that
\begin{enumerate}
	\item it is continuous on an interval \(I\)
	\item it is differentiable on \(I'=\) the interior of \(I\), and
	\item \(\forall x\in I'\quad f'(x)<0\)
\end{enumerate}
Then \(f\) is decreasing on \(I\)

\subsection{Increasing Differentiable Functions}
If \(f\) is differentiable on an open interval \(I\) and if \(f\) is increasing on \(I\), then
\[\forall x\in I\quad f'(x)\geq0\]

\subsection{Decreasing Differentiable Functions}
If \(f\) is differentiable on an open interval \(I\) and if \(f\) is decreasing on \(I\), then
\[\forall x\in I\quad f'(x)\leq0\]

\subsection{First Derivative Test}
If \(f\) is a function such that
\begin{enumerate}
	\item it is continuous at a critical point \(c\), and
	\item it is differentiable on an open interval containing \(c\), except at \(c\)
\end{enumerate}
Then if
\begin{enumerate}
	\item \(f'\) changes from negative to positive at \(c\)
	\begin{enumerate}
		\item then \(f\) has a local minimum value at \(c\)
	\end{enumerate}
	\item \(f'\) changes from positive to negative at \(c\)
	\begin{enumerate}
		\item then \(f\) has a local maximum value at \(c\)
	\end{enumerate}
	\item \(f'\) does not change sign at \(c\)
	\begin{enumerate}
		\item then \(f\) does not have a local extreme value at \(c\)
	\end{enumerate}
\end{enumerate}

\subsection{Second Derivative Test}
If \(f'(c)=0\), then if
\begin{enumerate}
	\item \(f''(c)>0\), then \(f\) has a local minimum value at \(c\)
	\item \(f''(c)<0\), then \(f\) has a local maximum value at \(c\)
\end{enumerate}

\subsection{Concavity}
\subsubsection{Concave Up}
If \(f\) is differentiable on an open interval \(I\). \(f\) is \textbf{concave up} if
\begin{enumerate}
	\item the graph of \(f\) lies above all its tangent lines on \(I\)
\end{enumerate}
More precisely,
\[\text{For all }a,b\in I,a\ne b\quad f(b)-f(a)>f'(a)(b-a)\]

\subsubsection{Concave Down}
If \(f\) is differentiable on an open interval \(I\). \(f\) is \textbf{concave down} if
\begin{enumerate}
	\item the graph of \(f\) lies below all its tangent lines on \(I\)
\end{enumerate}
More precisely,
\[\text{For all }a,b\in I,a\ne b\quad f(b)-f(a)<f'(a)(b-a)\]

\subsection{First Derivative and Concavity}
If \(f\) is a differentiable function on an open interval I

\subsubsection{Concave Up}
\[f'\text{ is increasing on I}\iff f\text{ is concave up on }I\]

\subsubsection{Concave Down}
\[f'\text{ is decreasing on I}\iff f\text{ is concave down on }I\]

\subsection{Concavity Test}
If \(f\) is twice differentiable on an open interval \(I\)

\subsubsection{Concave Up}
\[(\forall x\in I\quad f''(x)>0)\implies f\text{ is concave up on }I\]

\subsubsection{Concave Down}
\[(\forall x\in I\quad f''(x)<0)\implies f\text{ is concave down on }I\]

\subsection{Inflection Point}
\(f\) has an \textbf{inflection point} at \(c\) if
\begin{enumerate}
	\item\(f\) is continuous at \(c\), and
	\item\(f\) changes concavity at \(c\)
\end{enumerate}

\subsection{Twice Differentiable Inflection Point}
If a function \(f\)
\begin{enumerate}
	\item has an inflection point at \(c\), and
	\item is twice differentiable at \(c\)
\end{enumerate}
Then \(f''(c)=0\)

\subsection{L'H\^{o}pital's Rule (Baby Version)}
If \(f\) and \(g\) are functions differentiable at \(a\) such that
\begin{itemize}
	\item\(f(a)=g(a)=0\) and \(g'(a)\ne0\)
\end{itemize}
Then \(\lim\limits_{x\to a}\frac{f(x)}{g(x)}=\frac{f'(a)}{g'(a)}\)

\subsection{Cauchy's Mean Value Theorem}
If \(f\) and \(g\) are functions such that
\begin{enumerate}
	\item they are continuous on \([a,b]\)
	\item they are diferentiable on \((a,b)\), and
	\item\(\forall x\in(a,b)\quad g'(x)\ne0\)
\end{enumerate}
Then there exists \(c\in(a,b)\) such that
\[\frac{f'(c)}{g'(c)}=\frac{f(b)-f(a)}{g(b)-g(a)}\]

\subsection{L'H\^{o}pital's Rule (0/0 Version)}
If \(f\) and \(g\) are functions such that
\begin{enumerate}
	\item\(\lim\limits_{x\to a}f(x)=\lim\limits_{x\to a}g(x)=0\), and
	\item\(\lim\limits_{x\to a}\frac{f'(x)}{g'(x)}\) exists or equals \(\pm\infty\)
	\begin{enumerate}
		\item\(\frac{f'(x)}{g'(x)}\) is defined on an open interval \(I\) containing \(a\), except at \(a\)
		\begin{enumerate}
			\item\(f\) and \(g\) are differentiable on \(I\setminus\{a\}\)
			\item\(\forall x\in I\setminus\{a\}\quad g'(x)\ne0\)
		\end{enumerate}
	\end{enumerate}
\end{enumerate}
Then \(\lim\limits_{x\to a}\frac{f(x)}{g(x)}=\lim\limits_{x\to a}\frac{f'(x)}{g'(x)}\)

\subsubsection{Remarks}
\begin{enumerate}
	\item\(a\) may be finite or infinite
	\item\(x\to a\) may be replaced with one-sided limits
\end{enumerate}

\subsection{L'H\^{o}pital's Rule ($\infty$/$\infty$ Version)}
If \(f\) and \(g\) are functions such that
\begin{enumerate}
	\item\(\lim\limits_{x\to a}\abs{f(x)}=\lim\limits_{x\to a}\abs{g(x)}=\infty\), and
	\item\(\lim\limits_{x\to a}\frac{f'(x)}{g'(x)}\) exists or equals \(\pm\infty\)
	\begin{enumerate}
		\item\(\frac{f'(x)}{g'(x)}\) is defined on an open interval \(I\) containing \(a\), except at \(a\)
		\begin{enumerate}
			\item\(f\) and \(g\) are differentiable on \(I\setminus\{a\}\)
			\item\(\forall x\in I\setminus\{a\}\quad g'(x)\ne0\)
		\end{enumerate}
	\end{enumerate}
\end{enumerate}
Then \(\lim\limits_{x\to a}\frac{f(x)}{g(x)}=\lim\limits_{x\to a}\frac{f'(x)}{g'(x)}\)

\subsubsection{Remarks}
\begin{enumerate}
	\item\(a\) may be finite or infinite
	\item\(x\to a\) may be replaced with one-sided limits
	\item The condition \(\lim\limits_{x\to a}\abs{f(x)}\) is unnecessary
\end{enumerate}
\end{document}