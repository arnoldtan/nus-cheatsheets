\documentclass[../ma2002_notes.tex]{subfiles}
\begin{document}
\section{Techniques of Integration}
\subsection{Inverse Substitution Rule}
Let \(f\) be a continuous function. If \(x=g(t)\) is
\begin{enumerate}
	\item one-to-one, and
	\item\(g'\) is continuous
\end{enumerate}
Then
\[\int f(x)\,dx=\int f(g(t))g'(t)\,dt\]

\subsection{Integration by Parts}
Let \(u\) and \(v\) be functions that are differentiable with continuous derivatives. Then
\[\int\left(u\frac{dv}{dx}\right)\,dx=uv-\int\left(\frac{du}{dx}v\right)\,dx\]
Or in differential forms,
\[\int u\,dv=uv-\int v\,du\]

\subsection{Trigonometric Substitution}
If the integrand contains the square root of quadratic functions, one may try the \textbf{trigonometric substitution} method as follows:
\begin{enumerate}
	\item Complete the square to obtain the following forms:
\end{enumerate}
\begin{center}
\begin{tabular}{|c|c|c|c|}
\hline
Form & Substitution & Domain & Simplified Form\\
\hline
\(\sqrt{a^2-x^2}\quad(a>0)\) & \(x=a\sin t\) & \(t\in[-\frac{\pi}{2},\frac{\pi}{2}]\) & \(a\cos t\)\\[5pt]
\hline
\(\sqrt{a^2+x^2}\quad(a>0)\) & \(x=a\tan t\) & \(t\in(-\frac{\pi}{2},\frac{\pi}{2})\) & \(a\sec t\)\\[5pt]
\hline
\(\sqrt{x^2-a^2}\quad(a>0)\) & \(x=a\sec t\) & \(t\in[0,\frac{\pi}{2})\cup[\pi,\frac{3\pi}{2})\) & \(a\tan t\)\\[5pt]
\hline
\end{tabular}
\end{center}

\subsection{Unique Factorisation of Polynomial}
\textbf{Theorem.} Every non-constant single-variable polynomial with real coefficients can be uniquely factorised as the product of
\begin{enumerate}
	\item real linear factors
	\begin{itemize}
		\item For all \(a\in\mathbb{R},\ r\in\mathbb{N}\quad(x+a)^r\)
	\end{itemize}
	\item real irreducible quadratic factors
	\begin{itemize}
		\item For all \(b,c\in\mathbb{R},\ s\in\mathbb{N},\ b^2<4c\quad(x^2+bx+c)^s\)
	\end{itemize}
\end{enumerate}

\subsection{Proper Rational Function}
\textbf{Definition.} If \(f(x)=\frac{A(x)}{B(x)}\) is a rational function where \(A(x)\) and \(B(x)\) are polynomials. \(f(x)\) is a \textbf{proper rational function} if
\[\deg A(x)<\deg B(x)\]

\subsection{Converting from Improper to Proper Rational Function}
If \(f(x)=\frac{A(x)}{B(x)}\) where \(A(x)\) and \(B(x)\) are polynomials and \(\deg A(x)\geq\deg B(x)\). Then the following method converts \(f(x)\) into a proper rational function:
\begin{enumerate}
	\item Use long division to get
	\begin{itemize}
		\item\(A(x)=B(x)Q(x)+R(x)\), where \(\deg R(x)<\deg B(x)\)
	\end{itemize}
	\item Then \(f(x)=\frac{A(x)}{B(x)}=Q(x)+\frac{R(x)}{B(x)}\)
\end{enumerate}

\subsection{Decomposing Proper Rational Function into Partial Fractions}
\textbf{Theorem.} Every proper rational function can be uniquely expressed as the sum of \textbf{partial fractions}. Let \(f(x)=\frac{A(x)}{B(x)}\) be a proper rational function. Then \(f(x)\) is the sum of the following partial fractions:
\begin{enumerate}
	\item If \(x+a\) is a linear factor of \(B(x)\) with multiplicity \(r\)
	\begin{itemize}
		\item\(\displaystyle\frac{A_1}{x+a}+\frac{A_2}{(x+a)^2}+\ldots+\frac{A_r}{(x+a)^r}\)
	\end{itemize}
	\item If \(x^2+bx+c\) is an irreducible factor of \(B(x)\) with multiplicity \(s\)
	\begin{itemize}
		\item\(\displaystyle\frac{B_1x+C_1}{x^2+bx+c}+\frac{B_2x+C_2}{(x^2+bx+c)^2}+\ldots+\frac{B_sx+C_s}{(x^2+bx+c)^s}\)
	\end{itemize}
\end{enumerate}

\subsection{Integration of Rational Functions}
\subsubsection{Integral of $\frac{1}{(x+a)^k}$}
To evaluate \(\displaystyle\int\frac{dx}{(x+a)^k}\), use the substitution \(u=x+a\)

\subsubsection{Integral of $\frac{x}{(x^2+bx+c)^k}$}
To evaluate \(\displaystyle\int\frac{x}{(x^2+bx+c)^k}\,dx\), use the substitution \(u=x^2+bx+c\). Then
\[\int\frac{x}{(x^2+bx+c)^k}\,dx=\frac{1}{2}\int\frac{du}{u^k}-\frac{b}{2}\int\frac{dx}{(x^2+bx+c)^k}\]

\subsubsection{Integral of $\frac{1}{(x^2+bx+c)^k}$}
\begin{enumerate}
	\item\(x^2+bx+c=(x+\frac{b}{2})^2+(c-\frac{b^2}{4})\)
	\item Let \(u=x+\frac{b}{2},\ \alpha=\sqrt{c-\frac{b^2}{4}}\). Then \(x^2+bx+c=u^2+\alpha^2\)
	\begin{itemize}
		\item\(\displaystyle\int\frac{dx}{(x^2+bx+c)^k}=\int\frac{du}{(u^2+\alpha^2)^k}\)
	\end{itemize}
	\item Let \(u=\alpha v\). Then \(u^2+\alpha^2=\alpha^2(v^2+1)\)
	\begin{itemize}
		\item\(\displaystyle\int\frac{du}{(u^2+\alpha^2)^k}=\int\frac{\alpha\,dv}{(\alpha^2(1+v^2))^k}=\frac{1}{\alpha^{2k-1}}\int\frac{dv}{(1+v^2)^k}\)
	\end{itemize}
\end{enumerate}

\subsection{Universal Trigonometric Substitution}
Let \(f\) be a rational expression in two variables. Then
\[\int f(\sin x,\cos x)\,dx=\int f\left(\frac{2t}{1+t^2},\frac{1-t^2}{1+t^2}\right)\frac{2}{1+t^2}\,dt\]

\subsection{More Integral Formulae}
\subsubsection{Integral of Integer Power of $\frac{1}{1+x^2}$}
Let \(x=\tan t,\ t\in(-\frac{\pi}{2},\frac{\pi}{2})\)
\[\forall n\in\mathbb{N}\quad\int\frac{dx}{(1+x^2)^n}=\int(\cos t)^{2n-2}\,dt\]

\subsubsection{Integral of Non-Zero Power of $\cos x$}
\(\forall n\ne0\),
\[\int(\cos x)^n\,dx=\frac{1}{n}(\cos x)^{n-1}\sin x+\frac{n-1}{n}\int(\cos x)^{n-2}\,dx\]

\subsubsection{Integral of $\ln x$}
\[\int\ln x\,dx=x\ln x-x+C\]

\subsubsection{Integral of $\sin^{-1}x$}
\[\int\sin^{-1}x\,dx=x\sin^{-1}x+\sqrt{1-x^2}+C\]
\end{document}