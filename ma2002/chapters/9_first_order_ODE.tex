\documentclass[../ma2002_notes.tex]{subfiles}
\begin{document}
\section{First Order Ordinary Differential Equations}
\subsection{Ordinary Differential Equation}
\textbf{Definition.} An \textbf{ordinary differential equation} (abbr. \textbf{ODE}) is an equation of the form
\[F(x,y,\frac{dy}{dx},\ldots,\frac{d^ny}{dx^n})=0\]
where \(y\) is an implicit function in variable \(x\)

\subsubsection{Degree of ODE}
The highest order of the derivative is the \textbf{degree} of the ODE

\subsection{First Order ODE}
A \textbf{first order ODE} has the form
\[\frac{dy}{dx}=F(x,y)\]

\subsection{First Order ODE in only $x$}
\subsubsection{Method to Solve}
\(\displaystyle\frac{dy}{dx}=f(x)\) can be solved by simply integrating both sides w.r.t. \(x\). Then
\[y=\int f(x)\,dx\]

\subsection{First Order ODE in only $y$}
\subsubsection{Method to Solve}
\(\displaystyle\frac{dy}{dx}=g(y)\implies\frac{dx}{dy}=\frac{1}{g(y)}\ \)(for \(g(y)\ne0\)). Then,
\[x=\int\frac{1}{g(y)}\,dy\]

\subsection{Separable First Order ODE}
\textbf{Definition.} A first order ODE \(\displaystyle\frac{dy}{dx}=F(x,y)\) is separable if
\[F(x,y)=f(x)g(y)\]

\subsubsection{Method to Solve}
\(\displaystyle\frac{dy}{dx}=f(x)g(y)\implies\frac{1}{g(y)}\frac{dy}{dx}=f(x)\ \)(for \(g(y)\ne0\)). Then,
\[\int\frac{1}{g(y)}\,dy=\int f(x)\,dx\]

\subsection{Homogeneous Functions}
\textbf{Definition.} Let \(F(x_1,\ldots,x_m)\) be a function in \(m\) variables. \(F\) is \textbf{homogeneous of degree} \(\pmb{n}\) if \(\forall t\in\mathbb{R}\setminus\{0\}\)
\[F(tx_1,\ldots,tx_m)=t^nF(x_1,\ldots,x_m)\]

\subsection{Homogeneous First Order ODE}
\textbf{Definition.} A first order ODE \(\displaystyle\frac{dy}{dx}=F(x,y)\) is \textbf{homogeneous} if
\begin{enumerate}
	\item\(F(x,y)\) is homogeneous of degree zero
	\begin{itemize}
		\item\(\forall t\in\mathbb{R}\setminus\{0\}\quad F(tx,ty)=F(x,y)\)
		\item\textbf{Remarks.} The term \(\displaystyle\frac{y}{x}\) or \(\displaystyle\frac{x}{y}\) should appear in the function
	\end{itemize}
\end{enumerate}

\subsubsection{Method to Solve}
\begin{enumerate}
	\item Convert equation to standard form \(\displaystyle\frac{dy}{dx}=F(x,y)\)
	\item Let \(z=y/x\)
	\begin{itemize}
		\item\(y=xz\) and \(\displaystyle\frac{dy}{dx}=z+x\frac{dz}{dx}\)
		\item\(\forall x\ne0\quad F(x,y)=F(x,xz)=F(1,z)\)
	\end{itemize}
	\item The ODE becomes \(\displaystyle z+x\frac{dz}{dx}=F(1,z)\)
	\begin{itemize}
		\item This is separable in variables \(x\) and \(z\)
	\end{itemize}
\end{enumerate}

\subsection{First Order Linear ODE}
\textbf{Definition.} A first order ODE \(\displaystyle\frac{dy}{dx}=F(x,y)\) is \textbf{linear} if
\[F(x,y)=f(x)y+g(x)\]
The \textbf{standard form} of a first order linear ODE is
\[\frac{dy}{dx}+p(x)y=q(x)\]

\subsubsection{Method to Solve}
\begin{itemize}
	\item If \(\displaystyle\frac{dy}{dx}+p(x)y=0\). Then it is a separable equation.
	\item If \(\displaystyle\frac{dy}{dx}+p(x)y=q(x)\), proceed as follows:
	\begin{enumerate}
		\item Evaluate \(\int p(x)\,dx=P(x)+C\)
		\item Evaluate an \textbf{integrating factor} \(v(x)=e^{P(x)}\)
		\item\(\displaystyle y=\frac{1}{v(x)}\int v(x)q(x)\,dx\)
		\item\textbf{Remark.} Different integrating factors differ by a constant multiple and produce the same solution
	\end{enumerate}
\end{itemize}

\subsection{Bernoulli's Equation}
\textbf{Definition.} A \textbf{Bernoulli's differential equation} has the form
\begin{itemize}
	\item\(\displaystyle\frac{dy}{dx}+p(x)y=q(x)y^n\), where \(n\in\mathbb{R}\)
\end{itemize}

\subsubsection{Method to Solve}
\begin{itemize}
	\item If \(n=0\), it is a first order linear ODE
	\item If \(n=1\), it is a first order linear and separable ODE
	\item If \(n\ne0,1\), proceed as follows:
	\begin{enumerate}
		\item Let \(z=y^{1-n}\)
		\item\(\displaystyle\frac{dz}{dx}+(1-n)p(x)z=(1-n)q(x)\), which is a linear equation
	\end{enumerate}
\end{itemize}

\subsection{Exponential Growth and Decay}
\textbf{Theorem.} The general solution to \(\displaystyle\frac{dy}{dt}=ky\) is
\[y=C\exp(kt)\]

\subsection{Continuously Compounded Interest}
\textbf{Theorem.} If an amount \(A_0\) is invested at \(r\) interest. If the interest is compounded continuously, the value of the investment at time \(t\) is
\[A(t)=A_0\exp(rt)\]

\subsection{Exponential Decay}
Let \(\displaystyle\frac{dm}{dt}=km\), where \(k<0\). Then
\[m(t)=m(0)\exp(kt)\]

\subsubsection{Half-Life}
\textbf{half-life} \(\pmb{t_{1/2}}\) is the time required for half of the quantity to decay. Then
\[k=-\frac{\ln2}{t_{1/2}}\]

\subsection{Logistic Population Growth}
\textbf{Definition.} The \textbf{logistic growth} model with \(M>0\) as the \textbf{limiting population} or \textbf{carrying capacity} and \(r>0\) is given by
\[\frac{dP}{dt}=r(M-P)P\]

\subsubsection{Solving the Logistic Growth Model}
This is a Bernoulli's differential equation with the following general solution which is also known as the \textbf{logistic function}:
\[P(t)=\frac{M}{1+C\exp(-Mrt)}\]

\subsection{Newton's Law of Cooling}
Let \(T(t)\) be the temperature of an object at time \(t\) and \(T_S\) be the surrounding temperature. Then
\begin{itemize}
	\item\(\displaystyle\frac{dT}{dt}=-r(T-T_S)\), where \(r>0\) is a constant
\end{itemize}
Then
\[T(t)=T_S+(T_0-T_S)\exp(-rt)\]
\end{document}		