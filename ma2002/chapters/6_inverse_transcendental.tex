\documentclass[../ma2002_notes.tex]{subfiles}
\begin{document}
\section{Inverse Functions and Transcendental Functions}
\subsection{One-to-One Functions}
\textbf{Definition.} Let \(f\) be a function with domain \(D\). \(f\) is \textbf{one-to-one} if for any \(a,b\in D\),
\[a\ne b\implies f(a)\ne f(b)\]
Equivalently, for any \(a,b\in D\),
\[f(a)=f(b)\implies a=b\]

\subsection{Inverse Function}
\textbf{Definition.} Let \(f\) be a one-to-one function with domain \(A\) and range \(B\)
\begin{itemize}
	\item Then \(\forall y\in B\), there is a unique \(x\in A\) such that \(f(x)=y\)
\end{itemize}
The \textbf{inverse function} of \(f\), denoted by \(f^{-1}\), is defined by
\begin{itemize}
	\item For any \(x\in A\) and \(y\in B\quad f^{-1}(y)=x\iff y=f(x)\)
	\begin{enumerate}
		\item\(f^{-1}\) has domain \(B\) and range \(A\)
	\end{enumerate}
\end{itemize}

\subsection{Inverse of Inverse Function}
\[(f^{-1})^{-1}=f\]

\subsection{Composite of Function and its Inverse is an Identity Function}
If for any \(x\in A\) and \(y\in B\quad y=f(x)\iff x=f^{-1}(y)\), then
\begin{align*}
	&\forall x\in A\quad f^{-1}(f(x))=x\\
	&\forall y\in B\quad f(f^{-1}(y))=y
\end{align*}

\subsection{Equation for Inverse Functions}
Let \(f\) be a one-to-one function. Then \(f\) has an inverse function, say \(f^{-1}\). To find an equation for \(f^{-1}\),
\begin{enumerate}
	\item Let \(y=f(x)\)
	\item Solve for \(x\) in terms of \(y\):\(\quad x=f^{-1}(y)\)
	\item Interchange \(x\) and \(y\):\(\quad y=f^{-1}(x)\)
\end{enumerate}

\subsection{Geometric Meaning of Interchanging $x$ and $y$}
\begin{enumerate}
	\item In the Cartesian plane \(\mathbb{R}^2\), interchanging \(x\) and \(y\) has the same effect as reflecting the graph w.r.t. the straight line \(y=x\)
	\item Thus, the graph of \(f\) and \(f^{-1}\) are symmetric w.r.t. the line \(y=x\)
\end{enumerate}

\subsection{Continuous Functions are one-to-one iff monotonic}
\textbf{Theorem.} Let \(f\) be a function continuous on an interval \(I\), then
\begin{itemize}
	\item\(f\) is one-to-one \(\iff f\) is monotonic
	\begin{itemize}
		\item Monotonic means either increasing or decreasing
	\end{itemize}
\end{itemize}

\subsection{Inverse of Continuous Function is Continuous}
If \(f\) is a function one-to-one and continuous on an interval \(I\). Then the inverse function \(f^{-1}\) is also continuous

\subsection{Inverse of Differentiable Function is Differentiable}
If \(f\) is a function one-to-one and continuous on an interval \(I\). If \(f\) is differentiable at an interior point \(a\) of \(I\), \(f'(a)\ne0\), and \(b=f(a)\), then
\[(f^{-1})'(b)=\frac{1}{f'(a)}\]

\subsection{Inverse Trigonometric Functions}
\begin{center}
\begin{tabular}{|c|c|c|c|}
\hline
Function & Domain & Range & Derivative\\
\hline
\(\sin^{-1}x\) & \([-1,1]\) & \([-\frac{\pi}{2},\frac{\pi}{2}]\) & \(\displaystyle\frac{d}{dx}\sin^{-1}x=\frac{1}{\sqrt{1-x^2}}\)\\[10pt]
\hline
\(\cos^{-1}x\) & \([-1,1]\) & \([0,\pi]\) & \(\displaystyle\frac{d}{dx}\cos^{-1}x=-\frac{1}{\sqrt{1-x^2}}\)\\[10pt]
\hline
\(\tan^{-1}x\) & \(\mathbb{R}\) & \((-\frac{\pi}{2},\frac{\pi}{2})\) & \(\displaystyle\frac{d}{dx}\tan^{-1}x=\frac{1}{1+x^2}\)\\[10pt]
\hline
\(\cot^{-1}x\) & \(\mathbb{R}\) & \((0,\pi)\) & \(\displaystyle\frac{d}{dx}\cot^{-1}x=-\frac{1}{1+x^2}\)\\[10pt]
\hline
\(\sec^{-1}x\) & \((-\infty,-1]\cup[1,\infty)\) & \([0,\frac{\pi}{2})\cup[\pi,\frac{3\pi}{2})\) & \(\displaystyle\frac{d}{dx}\sec^{-1}x=\frac{1}{x\sqrt{x^2-1}}\)\\[10pt]
\hline
\(\csc^{-1}x\) & \((-\infty,-1]\cup[1,\infty)\) & \((0,\frac{\pi}{2}]\cup(\pi,\frac{3\pi}{2}]\) & \(\displaystyle\frac{d}{dx}\csc^{-1}x=-\frac{1}{x\sqrt{x^2-1}}\)\\[10pt]
\hline
\end{tabular}
\end{center}

\subsection{Inverse Trigonometric Identities}
\begin{align*}
	&\forall x\in[-1,1]\quad\sin^{-1}x+\cos^{-1}x=\frac{\pi}{2}\\
	&\forall x\in\mathbb{R}\quad\tan^{-1}x+\cot^{-1}x=\frac{\pi}{2}\\
	&\sec^{-1}x+\csc^{-1}x=\begin{cases}
		\frac{\pi}{2} & x\geq1\\
		\frac{5\pi}{2} & x\leq-1\\
	\end{cases}
\end{align*}

\subsection{Logarithmic Function}
\textbf{Definition.} The \textbf{natural logarithmic function} is defined by
\[\text{For }x>0\quad\ln x=\int_1^x\frac{1}{t}\,dt\]

\subsection{Properties of $\ln x$}
\begin{itemize}
	\item\(\ln1=0\)
	\item\(\ln x\) is continuous and differentiable on \(\mathbb{R}^+\)
	\item\(\forall x>0\quad\frac{d}{dx}\ln x=\frac{1}{x}\) and \(\frac{d^2}{dx^2}\ln x=-\frac{1}{x^2}\)
	\begin{itemize}
		\item\(\ln x\) is increasing and concave down on \(\mathbb{R}^+\)
	\end{itemize}
	\item\(\lim\limits_{x\to0^+}\ln x=-\infty\), \(\lim\limits_{x\to\infty}\ln x=\infty\), and range of \(\ln x\) is \(\mathbb{R}\)
\end{itemize}

\subsection{Logarithmic Laws}
\begin{align*}
	&\text{For }x>0,\ a>0\quad\ln(ax)=\ln a+\ln x\\
	&\text{For }x>0,\ r\in\mathbb{Q}\quad\ln(x^r)=r\ln x
\end{align*}

\subsection{Integral of $\frac{1}{x}$}
\textbf{Theorem.} \(\forall x\ne0\)
\[\frac{d}{dx}\ln\abs{x}=\frac{1}{x}\]

\subsection{Integral of Rational Trigonometric Functions}
\begin{align*}
	&\int\sec x\,dx=\ln\abs{\sec x+\tan x}+C\\
	&\int\csc x\,dx=-\ln\abs{\csc x+\cot x}+C\\
	&\int\tan x\,dx=-\ln\abs{\cos x}+C\\
	&\int\cot x\,dx=\ln\abs{\sin x}+C
\end{align*}

\subsection{Logarithmic Differentiation}
If \(y=[f_1(x)]^{r_1}\times\ldots\times[f_n(x)]^{r_n}\), where
\begin{enumerate}
	\item\(r_1,\ldots,r_n\in\mathbb{Q}\), and
	\item\(f_1,\ldots,f_n\) are non-zero differentiable functions
\end{enumerate}
Then \textbf{logarithmic differentiation} may be applied as follows:
\begin{enumerate}
	\item Take the absolute value:
	\begin{itemize}
		\item\(\abs{y}=\abs{f_1(x)}^{r_1}\times\ldots\times\abs{f_n(x)}^{r_n}\)
	\end{itemize}
	\item Take natural logarithm:
	\begin{itemize}
		\item\(\ln\abs{y}=r_1\ln\abs{f_1(x)}+\ldots+r_n\ln\abs{f_n(x)}\)
	\end{itemize}
	\item Differentiate w.r.t. \(x\):
	\begin{itemize}
		\item\(\displaystyle\frac{1}{y}\frac{dy}{dx}=\frac{r_1f_1'(x)}{f_1(x)}+\ldots+\frac{r_nf_n'(x)}{f_n(x)}\)
	\end{itemize}
\end{enumerate}

\subsubsection{Remarks}
Logarithmic differentiation is not applicable if \(y=0\)

\subsection{Euler's Number $e$}
\textbf{Definition.} The \textbf{Euler's number} \(\pmb{e}\) is the number such that \(\ln e=1\)

\subsection{Exponential Function $\exp x$}
\textbf{Definition.} Let \(\exp x=e^x\) be the inverse function of \(\ln x\). Then \(\exp x\) has domain \(\mathbb{R}\) and range \(\mathbb{R}^+\)

\subsection{Properties of $\exp x$}
\begin{enumerate}
	\item\(\lim\limits_{x\to-\infty}\exp x=0\) and \(\lim\limits_{x\to\infty}\exp x=\infty\)
\end{enumerate}

\subsection{Derivative of $\exp x$}
\[\frac{d}{dx}\exp x=\exp x\]

\subsection{Exponential Function $a^x$}
\textbf{Definition.} The exponential function of base \(a>0\) is defined by
\[\forall x\in\mathbb{R}\quad a^x=\exp(x\ln a)\]

\subsection{Properties of $a^x$}
For all \(a>0\) and \(x,y\in\mathbb{R}\)
\begin{align*}
	&\ln(a^x)=x\ln a\\
	&a^xa^y=a^{x+y}\\
	&a^{-x}=1/a^x\\
	&(a^x)^y=a^{xy}\\
	&\frac{d}{dx}a^x=a^x\ln a
\end{align*}

\subsection{Real Power Rule}
\textbf{Theorem.} \(\forall a\in\mathbb{R}\quad\forall x>0\),
\[\frac{d}{dx}x^a=ax^{a-1}\]

\subsection{Derivative of $x^x$}
\[\forall x>0\quad\frac{d}{dx}x^x=(\ln x+1)x^x\]

\subsection{Derivative of $f(x)^{g(x)}$}
Derivative of \(f(x)^{g(x)}\) can be found by logarithmic differentiation

\subsection{$e$ as a limit}
\textbf{Theorem.}
\[e=\lim_{x\to0}(1+x)^{1/x}\]

\subsection{Limits of $f(x)^{g(x)}$}
To find \(\lim\limits_{x\to a}f(x)^{g(x)}\), where \(f(x)>0\),
\begin{enumerate}
	\item Express \(f(x)^{g(x)}=\exp[g(x)\ln f(x)]\)
	\item Interchange \(\lim\) operator and \(\exp\) function
\end{enumerate}

\subsection{Hyperbolic Trigonometric Functions}
\subsubsection{Hyperbolic Sine Function}
\textbf{Definition.} The \textbf{hyperbolic sine function} is defined by
\[\sinh x=\frac{\exp x-\exp(-x)}{2}\]
\(\sinh x\) is increasing on its domain \(\mathbb{R}\) and has range \(\mathbb{R}\)

\subsubsection{Hyperbolic Cosine Function}
\textbf{Definition.} The \textbf{hyperbolic cosine function} is defined by
\[\cosh x=\frac{\exp x+\exp(-x)}{2}\]
\(\sinh x\) has domain \(\mathbb{R}\), it is increasing on \([0,\infty)\)

\subsection{Hyperbolic Trigonometric Identities}
\begin{align*}
	&\cosh^2t-\sinh^2t=1\\
	&\sinh(x+y)=\sinh x\cosh y+\cosh x\sinh y\\
	&\cosh(x+y)=\cosh x\cosh y+\sinh x\sinh y
\end{align*}

\subsection{Derivatives of Hyperbolic Trigonometric Functions}
\begin{align*}
	&\frac{d}{dx}\sinh x=\cosh x\\
	&\frac{d}{dx}\cosh x=\sinh x
\end{align*}

\subsection{Inverse Hyperbolic Trigonometric Functions}
\begin{center}
\begin{tabular}{|c|c|c|c|}
\hline
Function & Domain & Range & Derivative\\
\hline
\(\sinh^{-1}x\) & \(\mathbb{R}\) & \(\mathbb{R}\) & \(\displaystyle\frac{d}{dx}\sinh^{-1}x=\frac{1}{\sqrt{1+x^2}}\)\\[10pt]
\hline
\(\cosh^{-1}x\) & \([1,\infty)\) & \([0,\infty)\) & \(\displaystyle\frac{d}{dx}\sinh^{-1}x=\frac{1}{\sqrt{x^2-1}}\)\\[10pt]
\hline
\end{tabular}
\end{center}
\end{document}