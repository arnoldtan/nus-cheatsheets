\documentclass[../st2131_notes.tex]{subfiles}
\begin{document}
\chapter{Combinatorial Analysis}

\section{Multiplication Rule}
\subsection{Basic Principle of Counting}
Suppose that two experiments are to be performed. If
\begin{itemize}
	\item experiment 1 can result in any one of \(m\) possible outcomes; and
	\item experiment 2 can result in any one of \(n\) possible outcomes;
\end{itemize}
then together there are \(mn\) possible outcomes of the two experiments.

\subsection{Generalised Basic Principle of Counting}
Suppose that \(r\) experiments are to be performed. If
\begin{itemize}
	\item experiment 1 results in \(n_1\) possible outcomes;
	\item experiment 2 results in \(n_2\) possible outcomes;
	\item \(\ldots\)
	\item experiment \(r\) results in \(n_r\) possible outcomes;
\end{itemize}
then together there are \(n_1n_2\ldots n_r\) possible outcomes of the \(r\) experiments.

\section{Addition Rule}
Suppose that two experiments, experiment 1 and experiment 2, are mutually exclusive, i.e. either experiment 1 or experiment 2 occurs, but not both. If
\begin{itemize}
	\item experiment 1 can result in any one of \(m\) possible outcomes; and
	\item experiment 2 can result in any one of \(n\) possible outcomes;
\end{itemize}
then together there are \(m+n\) possible outcomes.

\section{Factorial}
\textbf{Definition.} \(n!=n(n-1)(n-2)\ldots3(2)(1)\) with the convention that \(0!=1\).

\section{Permutations}
\def\arraystretch{1.7}
\begin{tabular}{|p{2.5cm}|p{2.5cm}|p{6cm}|p{5.2cm}|}
\hline
\textbf{Number of Objects} & \textbf{Arrangment} & \textbf{Distinction Between Objects} & \textbf{Number of Permutations} \\
\hline
\multirow{4}{*}{\(n\)} & Linear & All Distinct & \(n!\) \\
\cline{2-4}
& Linear & \(n_1\) are alike, \(n_2\) are alike, \(\ldots\), \(n_r\) are alike & \(\displaystyle\frac{n!}{n_1!\,n_2!\,\ldots n_r!}\) \\
\cline{2-4}
& Circular & All Distinct & \(\displaystyle(n-1)!=\frac{n!}{n}\) \\
\cline{2-4}
& Necklace & All Distinct & \(\displaystyle\frac{(n-1)!}{2}\) \\
\hline
\end{tabular}

\section{Binomial Coefficient}
\textbf{Definition.} The \textbf{binominal coefficient} is
\[_nC_r=\binom{n}{r}=\frac{n!}{r!\,(n-r)!}\]
If \(n\) is a nonnegative integer, and \(r<0\) or \(r>n\), define
\[\binom{n}{r}=0\]

\section{Combinations}
If there are \(n\) distinct objects, of which we choose a group of \(r\) items, then the number of possible groups is given by \(\displaystyle\binom{n}{r}\).

\section{Properties of Binomial Coefficient}
\subsection{Symmetric Property}
For \(r=0,1,2,\ldots,n\),
\[\binom{n}{r}=\binom{n}{n-r}\]

\subsection{Only one way to choose nothing or everything}
\[\binom{n}{0}=\binom{n}{n}=1\]

\subsection{Recurrence Formula}
For \(1\leq r\leq n\),
\[\binom{n}{r}=\binom{n-1}{r-1}+\binom{n-1}{r}\]

\section{Binomial Theorem}
\textbf{Theorem.} Let \(n\) be a nonnegative integer, then
\[(x+y)^n=\sum_{k=0}^n\binom{n}{k}x^ky^{n-k}\]

\subsection{Special Case: $x=y=1$}
\[2^n=\sum_{k=0}^n\binom{n}{k}\]

\subsection{Special Case: $x=-1, y=1$}
\[0=\sum_{k=0}^n\binom{n}{k}(-1)^k\]

\subsection{Sum of Alternate Binomial Coefficients are Equal}
It follows from section 1.8.2 that
\[\binom{n}{0}+\binom{n}{2}+\binom{n}{4}+\ldots=\binom{n}{1}+\binom{n}{3}+\binom{n}{5}+\ldots\]

\section{Multinomial Coefficient}
\textbf{Definition.} The \textbf{multinominal coefficient} is
\[\binom{n}{n_1,n_2,\ldots,n_r}=\frac{n!}{n_1!\,n_2!\,\ldots n_r!}\]

\section{Divide $n$ Objects into $r$ Groups}
The number of ways to divide \(n\) objects into \(r\) distinct groups of size \(n_1,n_2,\ldots,n_r\) such that \(\sum\limits_{i=1}^rn_i=n\) is given by
\[\binom{n}{n_1}\binom{n-n_1}{n_2}\binom{n-n_1-n_2}{n_3}\ldots\binom{n-n_1-n_2-\ldots-n_{r-1}}{n_r}=\binom{n}{n_1,n_2,\ldots,n_r}\]

\section{Multinomial Theorem}
Let \(n\) be a nonnegative integer, then
\[(x_1+x_2+\ldots+x_r)^n=\sum_{n_1+n_2+\ldots+n_r=n}\binom{n}{n_1,n_2,\ldots,n_r}x_1^{n_1}x_2^{n_2}\ldots x_r^{n_r}\]

\section{The Number of Positive Integer Solutions of Equations}
Given the following equation in \(x_1, x_2, \ldots, x_r\),
\[x_1+x_2+\ldots+x_r=n\]
where \(x_1, x_2, \ldots, x_r\), and \(n\) are positive integers. The number of solutions to this equation is
\[\binom{n-1}{r-1}\]

\section{The Number of Nonnegative Integer Solutions of Equations}
Given the following equation in \(x_1, x_2, \ldots, x_r\),
\[x_1+x_2+\ldots+x_r=n\]
where \(x_1, x_2, \ldots, x_r\), and \(n\) are nonnegative integers. The number of solutions to this equation is
\[\binom{n+r-1}{r-1}\]
\end{document}